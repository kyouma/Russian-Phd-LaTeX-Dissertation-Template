\pdfbookmark{Общая характеристика работы}{characteristic}             % Закладка pdf
\section*{Общая характеристика работы}

\newcommand{\actuality}{\pdfbookmark[1]{Актуальность}{actuality}\underline{\textbf{\actualityTXT}}}
\newcommand{\progress}{\pdfbookmark[1]{Разработанность темы}{progress}\underline{\textbf{\progressTXT}}}
\newcommand{\aim}{\pdfbookmark[1]{Цели}{aim}\underline{{\textbf\aimTXT}}}
\newcommand{\tasks}{\pdfbookmark[1]{Задачи}{tasks}\underline{\textbf{\tasksTXT}}}
\newcommand{\aimtasks}{\pdfbookmark[1]{Цели и задачи}{aimtasks}\aimtasksTXT}
\newcommand{\novelty}{\pdfbookmark[1]{Научная новизна}{novelty}\underline{\textbf{\noveltyTXT}}}
\newcommand{\influence}{\pdfbookmark[1]{Практическая значимость}{influence}\underline{\textbf{\influenceTXT}}}
\newcommand{\methods}{\pdfbookmark[1]{Методология и методы исследования}{methods}\underline{\textbf{\methodsTXT}}}
\newcommand{\defpositions}{\pdfbookmark[1]{Основные положения, выносимые на защиту}{defpositions}\underline{\textbf{\defpositionsTXT}}}
\newcommand{\reliability}{\pdfbookmark[1]{Достоверность}{reliability}\underline{\textbf{\reliabilityTXT}}}
\newcommand{\probation}{\pdfbookmark[1]{Апробация}{probation}\underline{\textbf{\probationTXT}}}
\newcommand{\contribution}{\pdfbookmark[1]{Личный вклад}{contribution}\underline{\textbf{\contributionTXT}}}
\newcommand{\publications}{\pdfbookmark[1]{Публикации}{publications}\underline{\textbf{\publicationsTXT}}}
\newcommand{\structureandsize}{\pdfbookmark[1]{Структура и объем работы}{structureandsize}\underline{\textbf{\structureandsizeTXT}}}


{\actuality} 

В настоящее время в разных областях биологии и медицины повсеместно используются цифровые изображения и развиваются алгоритмы их обработки, в том числе и на основе глубокого обучения. Несмотря на быстрый научно"=технический прогресс, в силу различных причин оборудование, используемое для получения таких изображений, не всегда способно решить возникающие задачи, однако применение математических методов позволяет приблизиться к их решению и повысить качество уже полученных данных. Поэтому разработка алгоритмов повышения качества изображений с учётом современных условий остаётся актуальной.

Изображения клеточных структур, возникающие в области флуоресцентной микроскопии, являются одним из классов изображений, увеличение разрешения которых необходимо на практике. Одними из примеров таких задач являются ситуации, при которых разрешающая способность оптической системы микроскопа упирается в теоретический предел из-за дифракции света или когда увеличение оптической системы микроскопа не позволяет реализовать весь потенциал теоретической разрешающей способности.

Хотя появились методики преодоления дифракционного предела с использованием специального оборудования, на практике более доступным подходом является использование специальных стохастический мигающих красителей. Серия снимков, на которых молекулы красителя имеют разную яркость, преобразуется в изображение повышенного разрешения. Благодаря повышенному по сравнению с одним снимком количеству информации и слабой корреляции мигания молекул флуорофора, вычислительные методы позволяют существенно повысить разрешающие способности микроскопов без дорогостоящей модификации оборудования. В то же время из-за необходимости получать большое число изображений за короткий срок, повышается значимость таких параметров, как уровень шума и скорость выгорания красителей.

Для макроскопических же изображений в медицине по причине строгих ограничений со стороны закона и специфики области, связанной с повышенными рисками ошибок, важен вопрос аккуратной обработки изображений для избежания потери или привнесения информации в имеющиеся данные и контроля качества получаемых изображений. По-прежнему остаётся актуальной более стандартная задача повышения резкости. А для методов глубокого обучения в медицинской диагностике, которые внедряются в практику во всём мире, является анализ и предобработка входных данных. Необходим контроль соответствия входной информации и применяемого обученного или обучаемого алгоритма глубокого обучения.

При изучении рентгеновских снимков и, в частности, при диагностике туберкулёза важным фактором является жёсткость снимка, так как она напрямую влияет на его информативность: на изображении должны в достаточном количестве присутствовать важные для вынесения верного решения детали. Кроме того, необходима правильная настройка контрастности снимков, чтобы различия изображений из обучающего набора и тестовых снимков, которые получаются на различном оборудовании в разных условиях, не были велики настолько, что влияют на качество работы алгоритмов диагностики.

Основное внимание в данной научно"=квалификационной работе уделено повышению качества изображений клеточных структур и анализу качества медицинских изображений. Под качеством понимается как наличие у изображения определённых характеристик (резкость, контрастность, разрешение), так и их степень сходства с эталонными изображениями, для работы с которыми оптимизирован алгоритм машинного или глубокого обучения.

% {\progress}
% Этот раздел должен быть отдельным структурным элементом по
% ГОСТ, но он, как правило, включается в описание актуальности
% темы. Нужен он отдельным структурынм элемементом или нет ---
% смотрите другие диссертации вашего совета, скорее всего не нужен.

{\aim}

Цель данной работы состоит в разработке адаптивных методов анализа и обработки биомедицинских изображений различных модальностей на основе методов математического моделирования, их алгоритмическая и программная реализация для решения задач повышения разрешения изображений флуоресцентной микроскопии, повышения резкости и контроля качества медицинских изображений.


{\novelty}

В данной работе были разработаны:

\begin{enumerate}[beginpenalty=10000]
 	\item регуляризирующий метод повышения разрешения и резкости изображений флуоресцентной мигающей микроскопии;

	\item однопараметрический метод деформационного метода повышения резкости изображений;

	\item нейросетевой метод контроля качества рентгеновских снимков грудной клетки, основанный на визуальном определении жёсткости рентгенограммы;

	\item алгоритм классификации рентгенограмм грудной клетки для диагностики туберкулёза;
	
	\item алгоритм компьютерной диагностики туберкулёза по рентгеновскому снимку грудной клетки.
\end{enumerate}

{\influence}

Создан программный комплекс повышения разрешения изображений флуоресцентной мигающей микроскопии, повышения резкости медицинских изображений и определения качества рентгенограмм грудной клетки для задачи диагностики туберкулёза. Комплекс способен обрабатывать изображения раздела флуоресцентной микроскопии и различных медицинских областей.

Предложенные методы могут применяться как отдельно, так и в составе других систем анализа и повышения качества биомедицинских изображений, в качестве вспомогательных инструментов в работе учёных-биологов и врачей.

{\progress}
%TODO
TODO: Написать.
%Исследование, проведенное в данной диссертационной работе, затрагивает три различных области обработки и анализа изображений. При обработке изображений иммунофлюоресцентной микроскопии тканей кожи и анализе качетства снимков рентгенографии используется модель хребтовых структур. Понятие хребтовых структур для цифровых изображений было введено  Р.М.~Хараликом в 1983 году. Применение дескрипторов, основанных на хребтовых структурах, к анализу медицинских изображений было изучено в работах С.М.~Пайзера и его коллег. Позже в 1998 году Т. Линдеберг разработал метод детектирования $\gamma$-нормализованных хребтовых структур на основе локальной максимизации соответствующим образом нормализованных собственных значений матрицы Гессе. Данная методика позже была применены К.~Стегером для обнаружения дорог на карте, К.~Франги для сегментации кровеносных сосудов, а также для обнаружению криволинейных и трубчатых структур на снимках (Ю.~Сато).
%
%Специфика задач, рассматриваемых в данной диссертации, состоит в том, что рассматривается применение хребтовых структур для обнаружения межклеточных границ и костных структур. Обнаруженные при помощи анализа хребтовых структур особенности применяются в связке с алгоритмами машинного обучения.
%
%В данной диссертационной работе также рассмотрены задачи подавления шума на изображениях. Это одна из самых старых и наиболее исследованных областей обработки цифровых сигналов. К сожалению, проблема объективной оценки качества шумоподавления до сих пор полностью не решена. Здесь стоит обратить внимание на работы Ч.~Ванг и А.~Бовика, которые занимались разработкой индекса структурного сходства для сравнения и оценки качества изображений. В даной работе было предложено развитие идей индекса структурного сходства и разработан метод шумоподавления, основанный на модифицированной версии этого индекса.  


{\methods} 

В основе методологии исследования лежат методы математического моделирования в обработке и анализе изображений, ряд вычислительных экспериментов реализован в рамках задач машинного обучения и анализа изображений с помощью искусственных и реальных данных.

{\reliability}
%TODO
TODO: Написать.
%Достоверность результатов обеспечивается воспроизводимыми численными экспериментами на искусственных и реальных данных. При этом, большая часть наборов данных, на которых производилась обучение и тестирование моделей находятся в открытом доступе. Так же производилось сравнение результатов, полученных с помощью разработанных методов, и разметки, сделанной медицинскими специалистами. 

{\probation}
%TODO begin
TODO: Добавить из презентации к предзащите.
%Основные результаты работы докладывались на:
%
%\begin {enumerate}[beginpenalty=10000]
%	\item ХХХIII научно-практической конференции c международным участием, посвященной 170-летию со дня рождения А.И. Поспелова "<Рахмановские чтения: от дерматологии А.И. Поспелова до наших дней --- 170 лет"> (Москва, Россия, 2016);	
%
%	\item 26-ой международной конференции по компьютерной графике и зрению "<ГрафиКон'2016"> (Нижний Новгород, Россия, 2016);
%
%	\item ХХХIV научно-практической конференции c международным участием "<Рахмановские чтения"> (Москва, Россия, 2017);
%
%	\item 9-ой международной конференции по компьютерной графике и обработке изображений "<ICGIP"> (Циндао, Китай, 2017);
%
%	\item 28-ой международной конференции по компьютерной графике и зрению "<ГрафиКон'2018"> (Томск, Россия, 2018);
%
%	\item Научной конференции "<Тихоновские чтения 2021"> (Москва, Россия, 2021);
%
%	\item 6-ой международной конференции по биомедицинской визуализации и обработке сигналов "<ICBSP 2021"> (Сямэнь, Китай, 2021);				
%\end {enumerate}
%TODO end

\ifnumequal{\value{bibliosel}}{0}
{%%% Встроенная реализация с загрузкой файла через движок bibtex8. (При желании, внутри можно использовать обычные ссылки, наподобие `\cite{vakbib1,vakbib2}`).
    {\publications} 
    
    По теме исследования опубликовано~X работ, из них~X работы в изданиях, рекомендуемых для защиты в диссертационном совете МГУ имени М.В.~Ломоносова по специальности 05.13.18,
    и~X работа, опубликованная в сборниках трудов конференций.
     
}%
{%%% Реализация пакетом biblatex через движок biber
    \begin{refsection}[bl-author, bl-registered]
        % Это refsection=1.
        % Процитированные здесь работы:
        %  * подсчитываются, для автоматического составления фразы "Основные результаты ..."
        %  * попадают в авторскую библиографию, при usefootcite==0 и стиле `\insertbiblioauthor` или `\insertbiblioauthorgrouped`
        %  * нумеруются там в зависимости от порядка команд `\printbibliography` в этом разделе.
        %  * при использовании `\insertbiblioauthorgrouped`, порядок команд `\printbibliography` в нём должен быть тем же (см. biblio/biblatex.tex)
        %
        % Невидимый библиографический список для подсчёта количества публикаций:
        \printbibliography[heading=nobibheading, section=1, env=countauthorvak,          keyword=biblioauthorvak]%
        \printbibliography[heading=nobibheading, section=1, env=countauthorwos,          keyword=biblioauthorwos]%
        \printbibliography[heading=nobibheading, section=1, env=countauthorscopus,       keyword=biblioauthorscopus]%
        \printbibliography[heading=nobibheading, section=1, env=countauthorconf,         keyword=biblioauthorconf]%
        \printbibliography[heading=nobibheading, section=1, env=countauthorother,        keyword=biblioauthorother]%
        \printbibliography[heading=nobibheading, section=1, env=countregistered,         keyword=biblioregistered]%
        \printbibliography[heading=nobibheading, section=1, env=countauthorpatent,       keyword=biblioauthorpatent]%
        \printbibliography[heading=nobibheading, section=1, env=countauthorprogram,      keyword=biblioauthorprogram]%
        \printbibliography[heading=nobibheading, section=1, env=countauthor,             keyword=biblioauthor]%
        \printbibliography[heading=nobibheading, section=1, env=countauthorvakscopuswos, filter=vakscopuswos]%
        \printbibliography[heading=nobibheading, section=1, env=countauthorscopuswos,    filter=scopuswos]%
        %
        \nocite{*}%
        %
        {\publications}
        
    По теме исследования опубликовано~X работ, из них~X работы в журналах WoS, Scopus, RSCI, а также в изданиях, рекомендованных для защиты в диссертационном совете МГУ имени М.В.~Ломоносова по специальности 05.13.18 и~X работ, опубликованных в иных изданиях.
    	
    	%, 3 работы в журналах, входящих в перечень изданий ВАК при Минобрнауки России
        %откомментить при использовании автоподсчета
%        Основные результаты по теме диссертации изложены в~\arabic{citeauthor}~печатных изданиях,
%        \arabic{citeauthorvak} из которых изданы в журналах, рекомендованных ВАК\sloppy%
%        \ifnum \value{citeauthorscopuswos}>0%
%            , \arabic{citeauthorscopuswos} "--- в~периодических научных журналах, индексируемых Web of~Science и Scopus\sloppy%
%        \fi%
%        \ifnum \value{citeauthorconf}>0%
%            , \arabic{citeauthorconf} "--- в~тезисах докладов.
%        \else%
%            .
%        \fi%
%        \ifnum \value{citeregistered}=1%
%            \ifnum \value{citeauthorpatent}=1%
%                Зарегистрирован \arabic{citeauthorpatent} патент.
%            \fi%
%            \ifnum \value{citeauthorprogram}=1%
%                Зарегистрирована \arabic{citeauthorprogram} программа для ЭВМ.
%            \fi%
%        \fi%
%        \ifnum \value{citeregistered}>1%
%            Зарегистрированы\ %
%            \ifnum \value{citeauthorpatent}>0%
%            \formbytotal{citeauthorpatent}{патент}{}{а}{}\sloppy%
%            \ifnum \value{citeauthorprogram}=0 . \else \ и~\fi%
%            \fi%
%            \ifnum \value{citeauthorprogram}>0%
%            \formbytotal{citeauthorprogram}{программ}{а}{ы}{} для ЭВМ.
%            \fi%
%        \fi%
        
        
        % К публикациям, в которых излагаются основные научные результаты диссертации на соискание учёной
        % степени, в рецензируемых изданиях приравниваются патенты на изобретения, патенты (свидетельства) на
        % полезную модель, патенты на промышленный образец, патенты на селекционные достижения, свидетельства
        % на программу для электронных вычислительных машин, базу данных, топологию интегральных микросхем,
        % зарегистрированные в установленном порядке.(в ред. Постановления Правительства РФ от 21.04.2016 N 335)
    \end{refsection}%
    \begin{refsection}[bl-author, bl-registered]
        % Это refsection=2.
        % Процитированные здесь работы:
        %  * попадают в авторскую библиографию, при usefootcite==0 и стиле `\insertbiblioauthorimportant`.
        %  * ни на что не влияют в противном случае

    \end{refsection}%
        %
        % Всё, что вне этих двух refsection, это refsection=0,
        %  * для диссертации - это нормальные ссылки, попадающие в обычную библиографию
        %  * для автореферата:
        %     * при usefootcite==0, ссылка корректно сработает только для источника из `external.bib`. Для своих работ --- напечатает "[0]" (и даже Warning не вылезет).
        %     * при usefootcite==1, ссылка сработает нормально. В авторской библиографии будут только процитированные в refsection=0 работы.
}


{\contribution} 

Все результаты работы получены автором лично под научным руководством д.ф.-м.н., проф. А.С. Крылова. В работах, написанных в соавторстве, вклад автора диссертации в полученные результаты математического моделирования, численные методы и разработку комплекса программ является определяющим.

{\defpositions}

\begin {enumerate}[beginpenalty=10000]
	\item Метод повышения разрешения изображений мигающей флуоресцентной микроскопии, основанный на итерационном процессе максимизации регуляризирующего функционала.
	
	\item Однопараметрический метод повышения резкости медицинских изображений на основе деформации пиксельной сетки.
	
	\item Метод автоматического контроля качества рентгенограмм грудной клетки для задач машинного обучения, основанный на анализе и классификации уровня жёсткости рентгеновского снимка грудной клетки с помощью порядковой регрессии.
	
	\item Программный комплекс повышения разрешения изображений мигающей флуоресцентной микроскопии, повышения резкости медицинских изображений, определения качества рентгенограмм грудной клетки, набор данных для решения задачи компьютерной диагностики туберкулёза.
\end {enumerate}


 % Характеристика работы по структуре во введении и в автореферате не отличается (ГОСТ Р 7.0.11, пункты 5.3.1 и 9.2.1), потому её загружаем из одного и того же внешнего файла, предварительно задав форму выделения некоторым параметрам

%Диссертационная работа была выполнена при поддержке грантов \dots

\underline{\textbf{Структура и объем работы}} 

Работа состоит из~введения,
четырёх глав, заключения и списка литературы. Полный объем работы
73~страницы текста с~38~рисунками и~9~таблицами. Список
литературы содержит 71~наименование.

\pdfbookmark{Содержание работы}{description}                          % Закладка pdf
\section*{Содержание работы}

Во {\textbf{введении}} обосновывается актуальность работы, изложены её цель, научная новизна и практическая ценность, даны основные характеристики работы, сформулированы положения, выносимые на защиту, личный вклад автора, представлен отчёт об апробации работы и публикациях, содержащих основные результаты.

В {\textbf{первой}} главе рассматривается проблема автоматической сегментации изображений иммунофлюоресцентной микроскопии тканей кожи при кожных заболеваниях и выделения на них хребтовых структур. На основе структур, обнаруженных на каждом из сегментов изображения и руководствуясь размерами самих сегментов врач проводит диагностику пациента. В работе предлагается алгоритм автоматизации этого процесса. Рассмотрена задача улучшения качества диагностики пузырных дерматозов по изображениям тканей кожи (иммунофлюоресцентной микроскопии).  Для сегментации используются текстурные и статистические признаки. Предложен алгоритм детектирования и анализа структурных особенностей межклеточных границ на изображениях иимунофлюоресцентной микроскопии тканей кожи, который позволяет улучшить точность прогнозирования дальнейшего течения болезни. Для разработки методов используется база изображений, собранная в ГБУЗ МО МОНИКИ во время диагностики пациентов в отделении дерматологии.

В качестве алгоритма предобработки изображений иммунофлюоресцентной микроскопии тканей кожи предлагается метод, состоящий из выравнивания освещенности, медианной и гауссовской фильтрации.

Для выравнивания освещенности предлагается следующий алгоритм:

\begin{equation}
\begin{split}
 & L = G * I,\\
 &  G(x,y)=\frac{1}{2\pi{\sigma^2}}\exp{(-(x^2+ y^2 )/(2\sigma^2 ))}.\\
\end{split}
\end{equation}

 \noindent На сетке эти вычисления имеют вид:  
 
 \begin{equation}
\begin{split}
 & L[x,y]= \sum_{i,j=-n}^{n}G(i + n, j + n)I[x + i, y + j].\\
 & R=127.5\frac{I}{L + \epsilon}.\\
\end{split}
\end{equation} 

 \noindent Здесь $I$ --- исходное изображение, $G$ --- функция Гаусса, $\sigma$ --- среднеквадратическое отклонение, например, для изображений с разрешением около 500x500 используется $\sigma = 20$, $L$ --- изображение содержащее только низкочастотный сигнал, $R$ --- результирующее изображение, $n$ --- размер окна фильтрации в пикселях, $\epsilon$ --- малая величина, позволяющая избежать деления на 0 в случае темных изображений, в данной работе $\epsilon=0.5$.

Далее подробно описывается процесс выбора признаков для сегментации. В итоге для построения вектора признаков были выбраны статистические признаки:

\begin{itemize}
	\item Среднее значение: $MEAN = \frac{1}{N} \sum_{i}\sum_{j}I[i,j] $;
	\item Среднеквадратичное отклонение: \\
	$SD=\frac{1}{N} \sqrt{\sum_{i}\sum_{j} (I[i,j] - MEAN)^2} $;
	\item Вариация: $VAR= SD / MEAN$;
	\item Энергия: $ ASMN=\frac{1}{N}\sum_{i}\sum_{j}I[i,j]^2 $;
	\item Асимметрия: $SKEW= 1 / (N*SD^3) \sum_{i}\sum_{j} (I[i,j] - MEAN)^3 $;
	\item Эксцесс: $ KURT=1 / (N*SD^4) \sum_{i}\sum_{j} (I[i,j] - MEAN)^4 $;
	\item Энтропия: $ ENT=-\sum_{i}\sum_{j}I[i,j]\log_2{I[i,j]} $;	
\end{itemize}

\noindent где $N$ --- количество пикселей в блоке, по которому строится вектор признаков, суммирование по $i$, $j$ ведется по ширине и высоте блока в пикселях.

Также проводится анализ энергетических характеристик Лавса:

$$
\begin{aligned}
L5 (Level) =  [~1~4~6~4~1],\\
E5 (Edge) =  [-1 -2~0~2~1],\\
S5 (Spot) =  [-1~0~2~0 -1],\\
R5 (Ripple) =  [~1 -4~6 -4~1].\\
\end{aligned}
$$

Для вычисления двумерных масок одномерные вектора перемножаются между собой. В итоге получаются следующие маски: L5E5/E5L5,  L5R5/R5L5,  E5S5/S5E5,  S5S5,  R5R5,  L5S5/S5L5,  E5E5,  E5R5/R5E5,  S5R5/R5S5. Маски парные, суммарно их пятнадцать штук. Обычно симметричные пары комбинируются и усредняются. Таким образом, получается 9 масок. С помощью каждой из масок строится текстурная энергетическая карта.Для этого производится свертка изображения с каждой из этих масок. Каждая текстурная энергетическая карта является полноразмерным изображением --- результат обработки входного изображения с использованием конкретной маски. Для сегментации был использован многослойный персептрон.

На заключительном этапе на сегментированной области производилось выделение межклеточных границ изображения $I$ и особенностей в них (пример результата представлен на рисунке  \ref{fig:ml}) при помощи следующего алгоритма:

\begin {enumerate}
		\item	Находим вторые производные изображения $L = I * G$.
		\item	Для каждого пикселя составляем матрицу вида (аналитически вычисляются вторые производные функции Гаусса и изображение I сворачивается с ними): \[\begin{pmatrix} L_{xx} & L_{xy} \\ L_{xy} & L_{yy} \end{pmatrix}.\] 
		\item	Находим собственные значения матрицы $\lambda_1$,$\lambda_2$.
		\item	Вычисляем модуль отношения наименьшего и наибольшего по модулю собственного значения. Если оба собственных значения близки к $0$, то в данном месте нет особенности. Если модуль отношения наибольшего по модулю собственного значения к наименьшему больше некоторого порога $T$, то это хребтовая структура, иначе угол.
\end {enumerate}

\noindent Выбор порога $T$ адаптивно строится по набору изображений иммунофлюоресцентной микроскопии тканей кожи, аннотированных врачом. 

%\begin{figure}[ht]
%    \centerfloat{
%        \hfill
%        \subcaptionbox{Исходное изображение.}{%
%            \includegraphics[width=0.25\linewidth]{ml2a.png}}
%        \hfill
%        \subcaptionbox{Результат детектирования хребтовых структур без сегментации.}{%
%            \includegraphics[width=0.25\linewidth]{ml2b.png}}
%        \hfill
%        \hfill
%        \subcaptionbox{Результат детектирования хребтовых структур после сегментации.}{%
%            \includegraphics[width=0.25\linewidth]{ml2c.png}}
%        \hfill
%    }
%    \caption{Результаты сегментации и детектирования межклеточных границ для изображений иммунофлюоресцентной микроскопии тканей кожи}\label{fig:ml}
%\end{figure}

В заключительной части главы выполняется выделение связных компонент на изображении и удаление компонент малого радиуса. После чего проводится анализ полученных изображений при помощи математической модели оценки степени развития заболевания с использованием аппарата хребтовых структур. Метод прогнозирования течения болезни, основан на алгоритме подсчета межклеточных "<сопротивлений">. "<Сопротивление"> в точке зависит от интенсивности пикселя изображения в этой точке. Подсчет "<сопротивлений"> производится между центрами соседних клеток и усредняется для всего изображения.  Результатом работы данного метода является численное значение для каждого изображения, которое соответствует степени развития заболевания. Чем больше данное значение, тем более неблагоприятный прогноз. Неблагоприятное течение болезни называется торпидным. Результаты работы метода валидируются при помощи прогноза, данного врачом. Пример экспериментальных результатов применения метода приведен в таблице \ref{Table1}.

Исследование, проведенное в данной главе, показало применимость хребтовых структур для детектирования структурных особенностей на изображениях иммунофлюоресцентной микроскопии тканей кожи. Был разработан метод анализа структур на изображениях тканей кожи, включающий в себя алгоритм сегментации изображений иммунофлюоресцентной микроскопии тканей кожи и алгоритм детектирования структурных особенностей межклеточных границ на изображениях иммунофлюоресцентной микроскопии тканей кожи.

\begin{table}[ht!]
\centering
\begin{tabular}{|c|c|c|}
\hline Значение & Межклеточные & Прогноз \\
        "<сопротивления"> & структуры & течения болезни \\
\hline 7,32 & Пунктир, Сетка, Гранулы & Торпидный \\
\hline 4,25 & Сетка & Благоприятный \\
\hline 7,94 & Гранулы & Торпидный \\
\hline 4,95 & Сетка & Благоприятный \\
\hline 8,65 & Сетка, Гранулы & Торпидный \\
\hline 4,70 & Сетка & Благоприятный \\
\hline
\end{tabular}
\caption{Пример экспериментальных результатов применения метода прогнозирования течения болезни, основанного на алгоритме межклеточных "<сопротивлений">.} 
\label{Table1}
\end{table}    

Во {\textbf{второй}} главе рассматривается задача шумоподавления. Она является одной из самых старых задач повышения качества изображений, но, она по-прежнему актуальна. Методы подавления шума применяются к изображениям рентгена легких в качестве составной части комплексной системы диагностирования различных легочных заболеваний. Для построения диагностической системы требуется контроль качества исходного набора изображений, на котором будет производиться обучение алгоритма классификации. Поэтому рассматриваются различные методы оценки качества изображений. В главе разрабатываются методы оценки качества изображений рентгена легких для обучения и применения сверточных нейронных сетей. Выполняется разработка метрики оценки качества изображений на основе математического моделирования восприятия изображений человеческим глазом.

В начале главы рассматривается совершенствование алгоритма шумоподавления, основанного на самоподобии.

Самый известный из основанных на самоподобии алгоритмов шумоподавления --- алгоритм нелокального среднего (NLM).

\begin{equation}
\begin{split}
	&I_{NLM}(x,y) = \\
	&\frac{1}{W(x,y)}\sum_{x',y'\in\Omega(x,y)} w(x,y,x',y')I(x',y'),
\end{split}
\end{equation}

 \noindent где $W(x,y,x',y') = \sum_{x',y'\in\Omega(x,y)} w(x,y,x',y')$, $\Omega(x,y)$ --- окрестность точки $(x,y)$, в которой осуществляется сравнение блоков изображения, $w$ --- весовая функция, зависящая от схожести блоков с центрами в точках $(x,y)$ и $(x',y')$, $(x,y)$ --- точки изображения, в которых осуществляется применение алгоритма, чаще всего изображение обрабатывается целиком, за исключением граничных областей, в которых применяются особые техники (например, отзеркаливание граничных пикселей).

Размер окрестности $\Omega$ здесь может быть произвольным, в том числе и всем изображением.

\begin{equation}
\begin{split}
	&w(x,y,x',y') = \\
	&\exp(-\frac{\sum_{\xi,\eta\in\Omega_{patch}} (I(x+\xi,y+\eta) - I(x'+\xi,y'+\eta))^2}{2\rho^2}),
\end{split}
\end{equation} 
 
 \noindent где окрестность $\Omega_{patch}$ --- это фрагмент изображения вокруг точки, по которому и выполняется сравнение похожести двух точек, т.е. анализ схожести их текстур или блоков. Числитель в экспоненте является нормой $l_2$ разности двух векторов-параметров $\Omega_{patch}$, построенных как построчные выборки пикселей этих окрестностей. Веса алгоритма нелокального среднего зависят от евклидова расстояния между целыми блоками (фрагментами) вокруг соответствующих пикселей. 
 
NLM обеспечивает высокое качество получаемого изображения, но он имеет высокую вычислительную сложность. 

В главе прозводится обзор различных модификаций алгоритма нелокального среднего. Рассматриваются их достоинства и недостатки. Показано, что большую роль в качетве и эффективности работы алгоритма играет метрика, по которой считается расстояние между блоками. Отдельно рассматривается процедура отбора похожих блоков в алгоритме нелокального среднего.

Далее рассматриваются различные метрики и их эффективность при работе с алгоритмом нелокального среднего, на основе метрики SSIM предлагается новая модифицированная метрика структурного сходства (MSSIM):

\begin{equation}
\begin{split}
	&w(x,y,x',y') = \\
	&\Theta(T_{1}*\mu(x,y)\mu(x',y') - (\mu^{2}(x,y)+\mu^{2}(x',y')) \\
	&* \Theta(T_2*\sigma(x',y') - \sigma(x,y)) \\
	&* \Theta(s(x,y,x',y'))*f(s(x,y,x',y')).
\end{split}
\end{equation}

Данная метрика применяется в качестве весовой функции в алгоритме нелокального среднего. Здесь $\Theta$ --- ступенчатая единичная функция Хэвисайда. Функция $f$ служит для регулирования влияния слабо коррелированных блоков, в настоящей работе эта функция полагается линейной, $f(x)=x$ . Пороги $T_1$ и $T_2$ определяют допустимые пределы по отклонению яркости и контраста. 

В конце главы производится тестирование разработанного алгоритма на изображениях базы TID2013 и подбираются оптимальные параметры для разработанной метрики. В работе показано, что алгоритм шумоподавления основанный на модифицированном индексе структурного сходства является более универсальным алгоритмом, чем классический алгоритм нелокального среднего. Также возможно независимое применение модифицированного индекса структурного сходства в качестве метрики для оценки качества изображений.

В {\textbf{третьей}} главе описывается процедура создания сбалансированного набора изображений рентгена легких. Разрабатываются критерии первоначального отбора изображений в итоговую базу при помощи метода контроля качества рентгеновских снимков, строится математическая модель оценки качества снимков рентгена легких на основе числа видимых позвонков, ищется оптимальная структура разбиения по классам. Используется методика применения современных сверточных нейронных сетей для анализа изображений рентгена легких.

В начале главы из открытых источников формируется набор данных снимков рентгена легких и определяются типичные артефакты для данного вида снимков, такие как тепловой шум, освещенность и "<засветы">.

Решается задача предобработки рентгеновских снимков для подавления этих артефактов. 
 
Далее вводится понятие "<жесткости"> снимка и рассматривается влияние данного параметра на качество диагностики. В зависимости от  "<жесткости"> снимка на нём видны различные структурные особенности.

Для разработки автоматизированного метода контроля качества рентгеновских снимков формализуется эмпирический метод оценки  "<жесткости">, применяемый рентгенологами. Условным критерием оптимальной жесткости рентгеновских лучей для диагностики туберкулеза является отчетливая видимость только 3-4 верхних грудных позвонков. С целью определения видимых позвонков адаптируется автоматизированный метод нахождения позвоночника и выделения отдельных позвонков, ранее применявщийся на рентгенограммах позвоночника для определения сколиоза.

Область позвоночника находится при помощи нейросетевого метода выделения областей интереса с использованием прямоугольных областей. Использовались два набора данных, содержащие снимки рентгена легких различной  "<жесткости">. Врачами выполнена аннотация этих изображений, после чего объединенный набор данных был перемешан и разбит на тренировочный и тестовый в соотношении 80\% на 20\%. Проведено обучение модели нейросети resnet-50 и её тестирование. Результаты показали, что область позвоночника на изображениях рентгена легких локализуется вполне успешно.

После нахождения области позвоночника производится определение на изображении отдельных позвонков. Их обнаружение состоит из трех шагов:

\begin{enumerate}
\item Обнаружение центральной линии позвоночника;
\item Нахождение границ позвоночника;
\item Выделение отдельных позвонков.
\end{enumerate}

Процедура обнаружения центрольной линии позвоночника описывается формулой:

\begin{equation}
	s(x,y) =  \sum_{i=1}^{H} \sum_{j=-W/2}^{W/2} I(x +j, y + i),
	(x_{max}, y_{max}) = \argmax_{(x,y)}(s),
\end{equation}

 \noindent где $I$ --- область изображения, содержащая позвоночный столб с размерами $H_I$ и $W_I$, $H=H_I/8$, $W=W_I/4$, $(x,y)$ --- координаты окна ($x$ --- координата середины окна, $y$ --- координата нижней границы окна), $s(x,y)$ --- сумма значений интенсивностей пикселей в окне, а $(x_{max}, y_{max})$ --- координаты найденной точки.

На втором этапе используется два смежных окна одинакого размера. Эти два окна перемещаются не более чем на $r$ пикселей по нормали к центральной линии позвоночника в каждую из сторон от центра по оси $x$. Будем называть левой и правой границей позвоночника точки, находящиеся на границах позвонков (англ. edges), с координатами $(x_{l_i }, y_{l_i})$ и $(x_{r_i}, y_{r_i } ), i=1..n$, где $y_{l_i },y_{r_i }=y_{c_i }$ и $x_{l_i }<x_{r_i }$. Верхняя середина смежных окон, между которыми разность интенсивности максимальна считается точкой границы позвоночника. Процедура определения границ продолжается до тех пор, пока не будут исследованы все контрольные точки центральной линии позвоночника. 

На заключительном этапе обнаружения левой и правой границы позвоночника производится полиноминальная интерполяция по $n$ обнаруженным точкам с каждой стороны позвоночного столба. Здесь как и при обнаружении центральной линии позвоночника используются полиномы третьей степени. Параметры метода были привязаны к размерам обрабатываемых изображений $H_{I}$ и $W_{I}$ : $ H=H_{I} / 8, W=W_{I} / 4, p=H / 4, q=W / 2, r=W_{I} / 2$. 

Далее при использовании модифицированной карты хребтовых структур производится обнаружение отдельных позвонков. Карта строится по следующему алгоритму:

\begin {enumerate}
		\item	 Нахождение вторых производных изображения $L = I * G$.
		\item	 Составление попиксельной матрицы вида: \[\begin{pmatrix} L_{xx} & L_{xy} \\ L_{xy} & L_{yy} \end{pmatrix}. \]
		\item	 Нахождение собственных значения матрицы $\lambda_1$,$\lambda_2$. Построение 2-х карт хребтовых структур для значений  $\lambda_1$ и $\lambda_2$ --- $I_{1}$ и $I_{2}$.
		\item	 Бинаризация каждой из этих карт по порогу $\theta$. Итого имеем 2 карты хребтовых структур --- $I_{1_{\theta}}$ и $I_{2_{\theta}}$,  $I_{1,2_{\theta}} \in B\left\{ 0,1 \right\}$, $B\left\{ 0,1 \right\}$ --- бинарное пространство содержащее только 1 и 0.
		\item	 Нахождение модифицированной карты хребтовых структур: пересечение $I_{1_{\theta}}$ и $I_{2_{\theta}}$, где значение обеих бинарных карт равны 1. $I_{R} = I_{1_{\theta}} \cap I_{2_{\theta}} = \left\{ \begin{array} [c]{ll} 1,  I_{1_{\theta}}=1  \: \&\& \: I_{2_{\theta}}=1 \\ 0, \other \end{array} \right.$.
\end {enumerate}

Затем производится анализ числа независимых связных областей, у которых хотя бы одна из точек принадлежит центральной линии позвоночника. Результат --- количество найденных на изображении позвонков.
Итоговый общий вид алгоритма представлен на рисунке \ref{diagramm}.

%\begin{figure}[ht]
%\centering{\includegraphics[width=\linewidth]{diagram.png}}
%\caption{Общий вид алгоритма контроля качества рентгенограмм}
%\label{diagramm}
%\end{figure}

В продолжении главы производится валидация разработанного алгоритма контроля качества рентгеновских снимков. Для анализа соответствия по жесткости рентгенограмм из двух открытых наборов данных был сформирован единый набор. Использовался набор данных Montgomery (https://www.kaggle.com/datasets/raddar/tuberculosis-chest-xrays-montgomery), состоящий из 138 изображений в градациях серого с глубиной цвета 8 бит с разрешением~4000x4900 пикселей. 80 изображений --- это снимки здоровых пациентов, 58 снимков принадлежат больным туберкулезом. Второй используемый набор данных был Shenzhen (https://www.kaggle.com/datasets/raddar/tuberculosis-chest-xrays-shenzhen), который состоит из 662 изображений в градациях серого с глубиной цвета 8 бит с разрешением~3000x2900. В нем 326 изображений соответвуют здоровым пациентам, 336 снимков принадлежат больным туберкулезом. Для всех изображений есть клиническое описание диагностированных заболеваний. 

Для обоих наборов построены гистограммы по числу позвонков на снимках. Выполнено исследование применимости разработанного метода автоматического контроля качества рентгенограмм легких для улучшения качества классификации при помощи сверточных нейронных сетей.  Гипотеза состояла в том, что это может заметно повысить точность классификации данных на два класса --- снимки здоровых людей и снимки больных туберкулезом. Рассматривалось несколько пороговых значений количества позвонков.

Для решения задачи классификации в данном эксперименте была выбрана архитектура DenseNet. На первом этапе бралась модель DenseNet, предобученная на наборе данных ImageNet, заменялись классификационные слои сети полносвязным слоем с двумя выходами (поскольку рассматривалась бинарная классификация: люди, больные туберкулёзом, и здоровые) и проводилось дообучение модели. В этих экспериментах использовался оптимизатор стохастического градиентного спуска (SGD) с начальной скоростью обучения $10^{- 3}$ и импульсом (momentum) $0.9$, размер батча был 8.

Было проведено 4 типа экспериментов в зависимости от количества обнаруженных позвонков $k$, используемых в качестве порогового значения для фильтрации набора данных. Кроме того, для первых трех типов пороговых значений, рассматривалось 2 разных стратегии разделения набора данных:

\begin {enumerate}
		\item	 "<Однокомпонентная модель">. Здесь для обучения использовался отфильтрованный набор данных целиком.
		\item	 "<Двухкомпонентная модель">. Здесь отфильтрованный набор данных был разделен на два подмножества, соответствующих "<мягкому"> и "<жесткому"> уровню рентгеновского излучения. Две модели DenseNet обучались отдельно для каждого из подмножеств, а затем рассматривались как единая составная модель.
\end {enumerate}

\begin{table}[ht]
 \centerfloat{
\begin{tabular}{|l|l|l|l|l|}
\hline
\begin{tabular}[c]{@{}l@{}}Тип модели\end{tabular} & \begin{tabular}[c]{@{}l@{}}Фильтрация \\набора \\данных\end{tabular} & \begin{tabular}[c]{@{}l@{}}Размер \\набора \\данных\end{tabular} & \begin{tabular}[c]{@{}l@{}}Точность \\предсказания\end{tabular} \\ \hline
I                                                         & Нет ($k \geq 0$)                                                               & 5 386 (100\%)                                                            & 0.975    \\ \hline
II                                                        & Нет ($k \geq 0$)                                                               & 5 386 (100\%)                                                            & 0.979    \\ \hline
I                                                         & $k > 0$                                                                 		& 5 040 (94\%)                                                              & 0.984    \\ \hline
II                                                        & $k > 0$                                                                 		& 5 040 (94\%)                                                              & 0.987     \\ \hline
I                                                         & $3 \leq k \leq 8$                                                               & 3 925 (73\%)                                                            & 0.985    \\ \hline
II                                                        & $3 \leq k \leq 8$                                                               & 3 925 (73\%)                                                            & 0.988    \\ \hline
I                                                         & $5 \leq k \leq 7$.                                                               & 1 934 (36\%)                                                              & 0.966    \\ \hline
\end{tabular}
}
\caption{ Сравнительные результаты проведенных экспериментов по фильтрации набора данных с использованием предложенного метода автоматического контроля качества рентгенограмм легких.}
\label{Table2}
\end{table}

В таблице \ref{Table2} рассмотрены результаты бинарной классификации (туберкулез, здоровый) рентгенограмм легких. Размер набора данных после фильтрации приведен как по количеству изображений, так и в проценте от изначального объема данных. Точность предсказания рассчитывалась как отношение числа верных предсказаний к общему числу предсказаний.

Экспериментальные результаты показали, что фильтрация набора данных с использованием предложенного метода автоматического контроля качества рентгенограмм легких может заметно улучшить результаты классификации с применением сверточных нейронных сетей при диагностике туберкулеза с использованием рентгеновских снимков легких. Наилучшие результаты достигаются, когда в наборе данных оставлены только снимки с диапазоном видимых позвонков $3 \leq k \leq 8$ и используются двухкомпонентная модель. Разработанные модели классификаторов и метод контроля качества рентгеновских снимков были внедрены для апробации в ФГБУ "<НМИЦ ФПИ"> Минздрава России. 
 
{\textbf{Четвертая}} глава посвящена различным аспектам программной реализации алгоритмов, разработанных в предыдущих главах. Описывается разработанных программный комплекс, состоящий из трех различных программных модулей: 
\begin{itemize}
	\item Модуль обработки и анализа изображений иммунофлюоресцентной микроскопии тканей кожи.
	
	\item Модуль шумоподавления на основе модифицированного индекса структурного сходства.
	
	\item Модуль обработки и анализа ренгеновских снимков легких.
\end{itemize}

Представлен внешний вид интерфеса программных модулей. Программная реализация модулей выполнялась на языках C\#,  C++ и Python~3. Приведены технические характеристики ЭВМ, на которых выполнялось обучение и тестирование моделей глубокого обучения. 

\FloatBarrier
\pdfbookmark{Заключение}{conclusion}                                  % Закладка pdf
В \textbf{заключении} формулируются основные результаты работы.
%%% Согласно ГОСТ Р 7.0.11-2011:
%% 5.3.3 В заключении диссертации излагают итоги выполненного исследования, рекомендации, перспективы дальнейшей разработки темы.
%% 9.2.3 В заключении автореферата диссертации излагают итоги данного исследования, рекомендации и перспективы дальнейшей разработки темы.
\begin{enumerate}[beginpenalty=10000]
	\item \fixme{Разработан итерационный регуляризирующий алгоритм повышения разрешения и резкости изображений флуоресцентной мигающей микроскопии.}
	
	\item \fixme{Найдены оптимальные функции смещения для деформационного метода повышения резкости изображений для моделей трёх видов ядер размытия, возникающих на практике. Предложен малопараметрический вариант алгоритма.}
	
	\item \fixme{Разработан нейросетевой метод контроля качества рентгеновских снимков грудной клетки, основанный на автоматическом анализе жёсткости рентгенограммы. Показана возможность применения метода для повышения точности компьютерной диагностики туберкулёза лёгких.}
	
	\item \fixme{Реализован программный комплекс, состоящий из модулей повышения разрешения изображений флуоресцентной мигающей микроскопии, повышения резкости медицинских изображений, определения качества рентгенограмм грудной клетки для задачи диагностики туберкулёза лёгких.}
\end{enumerate}



\pdfbookmark{Литература}{bibliography}                                % Закладка pdf

\ifdefmacro{\microtypesetup}{\microtypesetup{protrusion=false}}{} % не рекомендуется применять пакет микротипографики к автоматически генерируемому списку литературы
\urlstyle{rm}                               % ссылки URL обычным шрифтом
\ifnumequal{\value{bibliosel}}{0}{% Встроенная реализация с загрузкой файла через движок bibtex8
    \renewcommand{\bibname}{\large \bibtitleauthor}
    \insertbiblioauthor           % Подключаем Bib-базы
    \nocite{*}
    %\insertbiblioexternal   % !!! bibtex не умеет работать с несколькими библиографиями !!!
}{% Реализация пакетом biblatex через движок biber
    % Цитирования.
    %  * Порядок перечисления определяет порядок в библиографии (только внутри подраздела, если `\insertbiblioauthorgrouped`).
    %  * Если не соблюдать порядок "как для \printbibliography", нумерация в `\insertbiblioauthor` будет кривой.
    %  * Если цитировать каждый источник отдельной командой --- найти некоторые ошибки будет проще.
    %
    \nocite{pchelintsev2019enhancement}
    \nocite{krylov2021regularization386034310}
    \nocite{krylov2017single}
    \nocite{krylov2018grid171613482}
    \nocite{dovganich2022automatic}
    \nocite{pchelintsev2023hardness}
%    \nocite{dovganich2016ridge}
%    \nocite{довганич2016иммунофлюоресцентная}
%    \nocite{довганич2016метод}
%    \nocite{довганич2017компьютерный}
%    \nocite{dovganich2018epidermis}
%    \nocite{довганич2018алгоритм}
%    \nocite{dovganich2019nonlocal}
%    \nocite{dogvanich2019dermatological}
%    \nocite{dovganich2021automatic}
    %\nocite{dovganich2022automatic}
     \insertbiblioauthor
%     \insertbiblioauthorgrouped

%    \ifnumgreater{\value{usefootcite}}{0}{
%        \begin{refcontext}[labelprefix={}]
%            \ifnum \value{bibgrouped}>0
%                \insertbiblioauthorgrouped    % Вывод всех работ автора, сгруппированных по источникам
%            \else
%                \insertbiblioauthor      % Вывод всех работ автора
%            \fi
%        \end{refcontext}
%    }{
%        \ifnum \totvalue{citeexternal}>0
%            \begin{refcontext}[labelprefix=A]
%                \ifnum \value{bibgrouped}>0
%                    \insertbiblioauthorgrouped    % Вывод всех работ автора, сгруппированных по источникам
%                \else
%                    \insertbiblioauthor      % Вывод всех работ автора
%                \fi
%            \end{refcontext}
%        \else
%            \ifnum \value{bibgrouped}>0
%                \insertbiblioauthorgrouped    % Вывод всех работ автора, сгруппированных по источникам
%            \else
%                \insertbiblioauthor      % Вывод всех работ автора
%            \fi
%        \fi
%        %  \insertbiblioauthorimportant  % Вывод наиболее значимых работ автора (определяется в файле characteristic во второй section)
%        \begin{refcontext}[labelprefix={}]
%            \insertbiblioexternal            % Вывод списка литературы, на которую ссылались в тексте автореферата
%        \end{refcontext}
%        % Невидимый библиографический список для подсчёта количества внешних публикаций
%        % Используется, чтобы убрать приставку "А" у работ автора, если в автореферате нет
%        % цитирований внешних источников.
%        \printbibliography[heading=nobibheading, section=0, env=countexternal, keyword=biblioexternal, resetnumbers=true]%
%    }
}
\ifdefmacro{\microtypesetup}{\microtypesetup{protrusion=true}}{}
\urlstyle{tt}                               % возвращаем установки шрифта ссылок URL
