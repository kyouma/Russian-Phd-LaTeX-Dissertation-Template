\chapter{Нейросетевой алгоритм контроля качества рентгеновских снимков грудной клетки в задаче диагностики туберкулёза лёгких}\label{ch:ch3}

Согласно клиническим рекомендациям Минздрава России, одним из первых этапов выявления и диагностики туберкулёза лёгких является рентгенофлюорографическое исследование~\cite{васильева2022туберкулез}. При этом в последние годы в практику внедряются компьютерные системы автоматической обработки медицинских данных и диагностики заболеваний на основе искуственного интеллекта, в том числе для в категории флюорографии~\cite{СМ106054, рг2023, рбк2023, интерфакс2023}.

Одной из актуальных задач при применении методов глубокого обучения в медицинской диагностике является анализ и предобработка входных данных. Необходим контроль соответствия входной информации и применяемого обученного (обучаемого) метода глубокого обучения. Это, например, требуется при контроле наличия адверсативных атак на данные~\cite{finlayson2019adversarial}.

Контроль качества рентгеновских снимков грудной клетки на основании разных параметров важен для полноценного анализа снимка и постановки правильного диагноза. Автоматическое определение контроля пространственных условий снимка (позы пациента, положения грудной клетки в кадре и т.п.) рассматривается в работах~\cite{nousiainen2021automating, von2020robust, jia2019application}. В работе~\cite{sadre2021validating} изучено влияние качества снимка на результаты автоматической диагностики COVID"~19.

В данной главе контроль входных рентгеновских снимков используется для проверки соответствия уровня облучения снимка уровню, требуемому для качественной диагностики туберкулёза лёгких. Рассматриваются:
\begin{enumerate}[beginpenalty=10000]
	\item задача автоматического определения уровня жёсткости рентгеновского снимка грудной клетки с помощью нейросетевого алгоритма;
	\item влияние предварительной фильтрации обучающей и валидационной выборок на качество работы алгоритма классификации в задаче диагностики туберкулёза лёгких по рентгеновским снимкам грудной клетки.
\end{enumerate}

Несмотря на то, что использование при анализе рентгенограммы грудной клетки, помимо фронтальной, также и боковой проекции повышает качество анализа, в том числе компьютерного, прирост качества для разных задач разный и далеко не всегда значительный~\cite{kluthke2016additional, hashir2020quantifying}. Кроме того, на практике в подавляющем числе случаев имеются только снимки, сделанные во фронтальной проекции~\cite{вишнякова2016частота}, поэтому в рамках данной главы рассматриваются фронтальные изображения грудной клетки.

\section{Метод оценки жёсткости рентгеновских снимков грудной клетки}

При изучении рентгеновских снимков грудной клетки, и в частности при диагностике туберкулёза лёгких, важным фактором является жёсткость снимка, так как она напрямую влияет на его информативность~\cite{chuiko1982effects, тимофеева2013основные}. Уровень жёсткости обусловлен дозой радиации, длиной волны излучения и особенностями тела пациента. При условии правильного контрастирования снимка уровень жёсткости можно определить визуально, подсчитав число отчётливо видимых на снимке верхних грудных позвонков: оптимальному уровню в задаче диагностики туберкулёза лёгких соответствуют 3"~4 видимых позвонка, а меньшее или большее число свидетельствует о том, что снимок слишком мягкий или жёсткий~\cite{тимофеева2013основные, сидоров2012методика}. Пример рентгеновского изображения позвоночного столба с указанием номеров позвонков приведён на Рис.~\ref{fig:vertebral-column}, а примеры снимков разного уровня жёсткости представлены на Рис.~\ref{fig:samples-different-hardness}.

\begin{figure}[ht]
	\centerfloat{
		\includegraphics[height=0.3\textheight]{example_vertebrae.png}
	}
	\caption{Пример рентгеновского изображения позвоночника с номерами шейных (C) и грудных (T) позвонков}
	\label{fig:vertebral-column}
\end{figure}

\begin{figure}[ht]
	\centerfloat{
		\hfill
		\subcaptionbox[List-of-Figures entry]{Мягкий снимок}{%
			\includegraphics[width=0.3\textwidth]{example_soft.png}}
		\hfill
		\subcaptionbox{Нормальный снимок}{%
			\includegraphics[width=0.3\textwidth]{example_normal.png}}
		\hfill
		\subcaptionbox{Жёсткий снимок}{%
			\includegraphics[width=0.3\textwidth]{example_hard.png}}
		\hfill
	}
	\caption{Примеры рентгеновских снимков грудной клетки разной жёсткости}
	\label{fig:samples-different-hardness}
\end{figure}

Настоящее исследование развивает идеи работы \cite{dovganich2022automatic}, где было показано, что оптимизация алгоритма диагностики заболеваний по рентгеновскому снимку грудной клетки для работы с изображениями близкой жёсткости вместе с автоматическим контролем качества рентгеновских изображений позволяет достичь точности классификации выше, чем у алгоритма, созданного в расчёте на обработку разнородных по уровню жёсткости изображений. В ходе выполнения данного исследования был применён полностью нейросетевой подход.

Предлагаемый метод оценки жёсткости включает 2 этапа: обучение и применение. В процессе обучения нейросетевая \fixme{модель}, являющаяся основой метода, однократно настраивается с целью оптимального решения задачи и сохраняется для дальнейшего использования. На этапе применения происходит непосредственный анализ изображений.

Процесс обучения выглядит следующим образом:
\begin{enumerate}[beginpenalty=10000]
	\item предобработка входных данных;
	\item итеративный процесс минимизации функции потерь (функционала, отображающего выходные значения нейросетевой \fixme{модели} в показатель их близости к оптимальным значениям для решения задачи) повторением шагов:
	\begin{enumerate}[beginpenalty=10000]
		\item подача предобработанных выходных данных на вход нейросетевой \fixme{модели} и получение её выходных значений;
		\item получение значения функции потерь для данных выходных значений \fixme{модели} и входных данных;
		\item шаг оптимизации;
		\item замер качества работы алгоритма;
		\item проверка условия ранней остановки процесса минимизации;
	\end{enumerate}
	\item сохранение \fixme{модели}.	
\end{enumerate}

Процесс применения метода состоит из следующих шагов:
\begin{enumerate}[beginpenalty=10000]
	\item предобработка входных данных;
	\item подача предобработанных выходных данных на вход нейросетевой \fixme{модели} и получение её выходных значений.
\end{enumerate}

\subsection{Использованные данные}

При решении рассматриваемой задачи были использованы 2 набора рентгенограмм грудной клетки, снятых во фронтальной проекции.

Первый набор изображений был собран в сотрудничестве с медиками из НПЦ~<<Фтизиатрия>> им. Е.Н.~Андреева в г. Якутск и применялся для обучения нейросетевой \fixme{модели} для метода определения жёсткости рентгенограмм.

Второй набор был сформирован из двух наиболее часто используемых при разработке методов компьютерной диагностики туберкулёза лёгких общедоступных наборов рентгенограмм, ставших, можно сказать, эталонными~\cite{oloko2022systematic, zeyu2022review, singh2022evolution, santosh2022advances}: Montgomery County~\cite{candemir2013lung} и Shenzhen~\cite{jaeger2013automatic}. Так как в большинстве работ наборы Montgomery County и Shenzhen используются вместе~\cite{oloko2022systematic, zeyu2022review}, то было принято решение их объединить и обозначать <<Montgomery"~Shenzhen>> (MC"~SZ). Полученный набор снимков использовался для оценки качества работы разработанного алгоритма на снимках, сделанных в других медицинских учреждениях, на другом оборудовании и при других условиях.

\subsubsection{Набор рентгенограмм грудной клетки SakhaTB} \label{subsubsec:dataset-yak-hardness}

Набор состоит из 1298 рентгеновских снимков грудной клетки больных туберкулёзом лёгких пациентов, собранных в нескольких медицинских учреждениях Республики Саха (Якутия) при помощи стационарных и переносных комплексов оборудования. Разрешение изображений разнится  и находится в диапазоне примерно от 2000x2000 до 3000x3000 пикселей. Большинство рентгенограмм имеет глубину цвета 16 бит, у остальных снимков она равна 8 битам.

Для дальнейшей работы глубина цвета снимков предварительно была приведена к 8 битам, а разрешение путём усреднения соседних пикселей было уменьшено в целое число раз до достижения значений около 1000x1000. Одновременно была проведена нормализация диапазона интенсивности пикселей с отсечением (англ.~clipping) по 0.5\% пикселей с каждого конца гистограммы изображения и обрезкой по 1.5\% пикселей с каждой стороны изображений, чтобы увеличить долю диапазона допустимых значений 8 бит информации, приходящуюся на внутренние органы грудной клетки. Примеры снимков приведены на Рис.~\ref{fig:samples-yak-hardness}.

Подмножество снимков этого набора было выложено вместе с рентгенограммами грудной клетки здоровых пациентов для использования в открытый доступ под названием Sakha"~TB (см.~п.~\ref{sec:sakha-tb}).

\begin{figure}[ht]
	\centerfloat{
    	\includegraphics[width=0.9\textwidth]{sakhatb.png}
	}
	\caption{Примеры изображений из набора SakhaTB}
	\label{fig:samples-yak-hardness}
\end{figure}

%\subsubsection{Набор Montgomery County} \label{subsubsec:dataset-mc}

\subsubsection{Набор рентгенограмм грудной клетки Montgomery"~Shenzhen} \label{subsubsec:dataset-mc-sz}

Набор изображений Montgomery County (далее MC) предоставляется Национальной библиотекой медицины Национальных институтов здравоохранения США и собран Министерством здравоохранения и социальных служб США в округе Монтгомери штата Мэриленд. Он содержит рентгеновские снимки грудной клетки в оттенках серого с глубиной цвета 8 бит в формате PNG с разрешением 4020x4892 пикселей и их метки соответствия 2 классам: 80 рентгенограмм здоровых и 58 рентгенограмм больных туберкулёзом лёгких пациентов. %Примеры снимков приведены на Рис.~\ref{fig:samples-mc-sz}.

%\subsubsection{Набор Shenzhen} \label{subsubsec:dataset-sz}

Набор изображений Shenzhen (далее SZ) предоставляется Национальной библиотекой медицины Национальных институтов здравоохранения США и собран Медицинским колледжем провинции Гуандун в Народном госпитале №~3 г.~Шэньчжэнь в КНР. Он содержит рентгеновские снимки грудной клетки в оттенках серого с глубиной цвета 8 бит в формате PNG с различным разрешением (примерно 3000x3000 пикселей) и их метки соответствия 2 классам: 326 рентгенограмм здоровых и 336 рентгенограмм больных туберкулёзом лёгких пациентов.

Примеры снимков объединённого набор приведены на Рис.~\ref{fig:samples-mc-sz}.

%\begin{figure}[ht]
%	\begin{minipage}[b][][b]{0.9\textwidth}\centering
%		\includegraphics[width=0.9\textwidth]{mcsz-normal.png} \\
%		{а)~Здоровые пациенты}
%	\end{minipage}
%	\hfill
%	\begin{minipage}[b][][b]{0.9\textwidth}\centering
%		\includegraphics[width=0.9\textwidth]{mcsz-tb.png} \\
%		{б)~Пациенты, больные туберкулёзом}
%	\end{minipage}
%	\caption{Примеры изображений из объединённого набора Montgomery"~Shenzhen (MC"~SZ)}
%	\label{fig:samples-mc-sz}
%\end{figure}

\begin{figure}[ht]
	\centerfloat{
%		\hfill
		\subcaptionbox[List-of-Figures entry]{Здоровые пациенты}{%
			\includegraphics[height=0.25\textheight]{mcsz-normal.png}}
%		\hfill
		\vfill
%		\hfill
		\subcaptionbox{Пациенты, больные туберкулёзом лёгких}{%
			\includegraphics[height=0.25\textheight]{mcsz-tb.png}}
%		\hfill
		\vfill
	}
	\caption{Примеры изображений из объединённого набора Montgomery"~Shenzhen (MC"~SZ)}
	\label{fig:samples-mc-sz}
\end{figure}

\subsubsection{Аннотирование данных}

Для обоих наборов (SakhaTB, собранный в рамках исследования, и MC"~SZ) была проведена процедура аннотации врачом-рентгенологом из НПЦ~<<Фтизиатрия>> им. Е.Н.~Андреева в г. Якутск. В рамках процедуры каждому изображению ставился в соответствие его уровень жёсткости, выраженный числом отчётливо видимых на этом снимке верхних грудных позвонков.  Распределение изображений по такому показателю жёсткости представлено на Рис.~\ref{fig:vertebrae-yak-hardness}"~\ref{fig:vertebrae-mc-sz}.

\begin{figure}[ht]
	\centerfloat{
		\includegraphics[height=0.3\textheight]{total_vertebrae_hist.png}
	}
	\caption{Гистограмма распределения снимков набора SakhaTB по числу отчётливо видимых позвонков}
	\label{fig:vertebrae-yak-hardness}
\end{figure}

\begin{figure}[ht]
	\centerfloat{
		\includegraphics[height=0.25\textheight]{total_vertebrae_hist_mc-sz.png}
	}
	\caption{Гистограмма распределения снимков набора MC"~SZ по числу отчётливо видимых позвонков}
	\label{fig:vertebrae-mc-sz}
\end{figure}

\subsection{Адаптивная предобработка входных изображений} \label{subsec:tb-hardness-preprocessing}

Хотя финальным критерием при определении уровня жёсткости рентгеновского снимка грудной клетки является число чётко контурируемых верхних грудных позвонков~\cite{тимофеева2013основные, сидоров2012методика}, для определения уровня контраста перед исследованием требуется обращать внимание на видимость других областей грудной клетки, (например, на элементы лёгочного рисунка) и органов~\cite{сидоров2012методика}. На основании этого было принято решение не ограничивать область исследования на снимках пределами грудной клетки, а рассматривать изображения целиком.

Так как условия получения снимка, характеристики тела пациента и принципы считывания сигнала и хранения информации каждого конкретного устройства могут сильно влиять на характер рентгеновских изображений, перед подачей их на вход алгоритма определения жёсткости выполнялся этап предобработки, который заключается в последовательности следующих шагов:
\begin{enumerate}[beginpenalty=10000]
	\item автоматическое контрастирование изображения:
	\begin{equation}
	h \left( x \right) = 255 \cdot \frac{x - p_{0.5}}{p_{99.5} - p_{0.5}}, \nonumber
	\end{equation}
	где $x$ "--- значение интенсивности пикселя входного изображения, $p_{0.5}$ и $p_{99.5}$ "--- это 0.5\%"~ и 99.5\%"~процентили значений интенсивности всех пикселей изображения;
	\item автоматическая гамма-коррекция интенсивности пикселей изображения:
	\begin{equation}
	g \left( x \right) = 255 \cdot {\left( \frac{x}{255} \right)}^{\gamma}, \quad \gamma = \log_{\mu / 255}{0.5}, \nonumber
	\end{equation}
	где $\mu$ "--- средняя интенсивность всего изображения;
	\item уменьшение изображения до входного разрешения, используемого нейронной сетью (512х512 пикселей для \fixme{модели} ResNet"~18 и 384x384 пикселей для EfficientNetV2"~S).
	\item опциональный шаг глобальной или локальной (CLAHE~\cite{pizer1987adaptive}) эквализации гистограммы. Размер стороны квадратного окна  в пикселях, используемого в методе локальной эквализации гистограммы, прямо зависит от размера стороны изображения и составляет $\frac{1}{2^n}$ от него, где $n\in\mathbb{N}$ "--- параметр метода. Влияние наличия этого шага и размера окна на качество работы алгоритма будет показано ниже.
\end{enumerate}

\subsection{Метод определения жёсткости рентгеновских снимков грудной клетки}

Так как уровень жёсткости рентгеновского снимка "--- это упорядоченная величина, для сохранения отношений упорядоченности между классами задача его определения рассматривалась как задача порядковой регрессии (также иногда называется порядковой классификацией)~\cite{7161338, 353a0d24-9c24-3a11-a330-afc86b9c39c8}.

Для решения задачи автоматического определения жёсткости рентгеновского снимка грудной клетки был разработан нейросетевой метод. На вход алгоритма поступает рентгеновский снимок грудной клетки и подвергается предобработке, затем при помощи нейронной сети ему ставится в соответствие вещественное число на отрезке $\left[0, 1\right]$, которое является внутренним безразмерным показателем жёсткости снимка, на основании которого после сравнения с настраиваемыми порогами изображение относится к одному из рассматриваемых классов жёсткости.  Пороги являются частью \fixme{модели} и настраиваются вместе с весами слоёв нейронной сети в процессе обучения. Преимуществом такого подхода является то, что с помощью внутреннего показателя жёсткости можно ранжировать изображения относительно друг друга даже в том случае, когда обрабатываемое изображение значительно отличается от обучающей выборки и для такого снимка пороги разделения классов жёсткости могут быть настроены неверно.

Для решения задач обработки медицинских изображений и, в частности, диагностики заболеваний широко используются~\cite{oloko2022systematic} свёрточные нейронные сети семейств ResNet~\cite{he2016deep}, DenseNet~\cite{huang2017densely} и др.

В ходе исследования для решения задачи определения жёсткости снимков, вследствие малого размера выборки, была использована компактная сеть ResNet"~18 с меньшим числом параметров и, следовательно, меньшей склонностью к переобучению по сравнению с более крупными сетями этой же архитектуры или представителями других вышеупомянутых архитектур.

Также было проведено сравнение качества решения этой задачи с таковым у компактной сети более новой архитектуры свёрточных нейронных сетей EfficientNetV2"~S~\cite{tan2021efficientnetv2}, так как в задачах классификации изображений она проявляет себя лучше перечисленных выше. Её основное отличие от них заключается в оптимизации работы свёрточных слоёв с помощью их пропорционального масштабирования, смены порядка операций разной размерности, уменьшения размера ядер свёртки и отказа от «тяжёлых» слоёв, что позволяет уменьшить потребление памяти и задействовать освободившиеся ресурсы на увеличение глубины и обобщающей способности нейронной сети.

\subsection{Эксперименты и результаты} \label{subsec:tb-hardness-experiments}

Как видно из Рис.~\ref{fig:vertebrae-yak-hardness}, набор изображений значительно несбалансирован и для ряда подуровней жёсткости содержит крайне мало примеров. Исходя из этого было принято решение, используя определённое врачом-рентгенологом число отчётливо видимых на снимках верхних грудных позвонков, разделить все снимки по уровню жёсткости на 3 группы, ориентируясь на медицинские критерии: мягкие (видно менее 3 позвонков), нормальные (видно 3"~4 позвонка) и жёсткие (видно более 4 позвонков)~\cite{тимофеева2013основные, сидоров2012методика}. Количество снимков в сформированных группах представлено на Рис.~\ref{fig:hardness-yak-hardness}. Именно эти три класса рассматривались в качестве допустимых значений целевой переменной для задачи порядковой регрессии.

\begin{figure}[ht]
	\centerfloat{
		\includegraphics[height=0.25\textheight]{total_hardness_hist.png}
	}
	\caption{Гистограмма распределения снимков набора SakhaTB по уровню жёсткости}
	\label{fig:hardness-yak-hardness}
\end{figure}

Для ускорения процесса обучения за счёт наличия готовых низкоуровневых фильтров при решении обеих задач в качестве начального состояния нейронной сети использовались веса соответствующей \fixme{модели}, обученной на решение задачи классификации изображений реального мира ImageNet"~1K~\cite{russakovsky2015imagenet}. Последний слой был заменён на полносвязный слой с 1 выходом и 2 настраиваемыми в процессе обучения параметрами"=порогами пороговой \fixme{модели} порядковой регрессии~\cite{rennie2005loss}, во время дальнейшего обучения были задействованы все слои.

Все базовые наборы изображений и их рассмотренные комбинации делились на обучающую, валидационную и тестовую выборки в соотношении 64:16:20 с предварительным случайным перемешиванием изображений и стратификацией по классу жёсткости для сохранения пропорций между классами. Изображения обучающей выборки в процессе обучения последовательно подвергались четырём случайным преобразованиям, а именно:
\begin{enumerate}[beginpenalty=10000]
	\item поворотам (в пределах 15 градусов в каждую сторону),
	\item масштабированию (коэффициент выбирался случайно из отрезка $\left[ 0.8, 1.2 \right]$,
	\item сдвигам (до 30\% размера изображения по каждой оси),
	\item изменению яркости и контраста (до 20\% в каждую сторону).
\end{enumerate}

В качестве функции потерь в процессе обучения и валидации была выбрана функция потерь пороговой порядковой регрессии в виде суммы слагаемых, число которых зависит от числа классов (all"=threshold)~\cite{rennie2005loss}:

\begin{equation}
	\begin{cases}
		L \left( z, y \right) = \sum_{k=1}^{K-1} f \left( s \left( k, y \right) \cdot \left( \theta_k-z \right) \right), \\
		s \left( k, y \right) =
		\begin{cases}
			-1, k < y, \\
			+1, \ k \geq y,
		\end{cases}
	\end{cases} \nonumber
\end{equation}

\noindent где $z$ "--- выход нейронной сети (безразмерный показатель жёсткости), принимающий значения из отрезка $\left[ 0, 1 \right]$ (чем ближе к $1$, тем выше жёсткость снимка), $y$ "--- истинный класс соответствующего этому выходу изображения, $K$ "--- общее число классов, $\theta_1 < \theta_2 < \ldots < \theta_{K-1}$ "--- пороги, делящие действительную прямую на $K$ частей, а $f(x)$ "--- базовая функция потерь бинарной классификации, в качестве которой использовалась логистическая функция потерь:

\begin{equation}
	f \left( x \right) = \ln{\frac{1}{1+e^{-x}}}. \nonumber
\end{equation}

Так как даже после сведения количества классов в задаче порядковой регрессии к трём выборка всё равно осталась сильно несбалансированной, то применялось взвешивание значений функции потерь для каждого примера с весами, обратно пропорциональными количеству изображений соответствующего класса. В роли показателя качества порядковой регрессии на этапе валидации выступала сбалансированная по классам средняя абсолютная ошибка (macro"=averaged MAE, далее mMAE)~\cite{baccianella2009evaluation}.

Также в целях получения базовой оценки качества решения задачи была обучена \fixme{модель} для обыкновенной классификации изображений на 3 класса. Последний слой \fixme{модели}"=основы был заменён на полносвязный с 3 выходами. Роль функции потерь выполняла кросс-энтропия:

\begin{equation}
	\begin{cases}
		CE \left( z,y \right) = \sum_{k=1}^{K}{\mathbb{I} \left[ y = k \right] \cdot \ln{\left( softmax \left( z \right)_k \right)}}, \\
		softmax \left( z \right)_k = \frac{e^{z_k}}{\sum_{i=1}^{K}e^{z_i}},\quad \mathbb{I} \left[ y = k \right] =
		\begin{cases}
			1, y = k, \\
			0, y \neq k,
		\end{cases}
	\end{cases} \nonumber
\end{equation}

\noindent а роль показателя качества на этапе валидации "--- сбалансированная точность (далее BalAcc)~\cite{brodersen2010balanced}.

Для оптимизации функции потерь применялся алгоритм градиентного спуска AdamW~\cite{loshchilov2018decoupled} с параметрами $lr = 5 \cdot {10}^{-6}$, $\beta_1 = 0.9$, $\beta_2 = 0.999$, $\lambda = 0.01$. Размер пакета (англ.~batch) изображений был равен 64 для \fixme{модели} на основе ResNet"~18 и 16 для \fixme{модели} на основе EfficientNetV2"~S; в обоих случаях применялось накопление градиента на протяжении 8 и 2 итераций соответственно (до достижения размера <<виртуального пакета>> в 128 объектов). В конце каждой эпохи качество \fixme{модели} замерялось на валидационной выборке; при отсутствии уменьшения значения функции потерь на ней в течение 10 эпох шаг градиентного спуска уменьшался в 5 раз, а при отсутствии улучшения в течение 31 эпохи обучение прекращалось. Переобучение контролировалось замерами функции потерь и вышеуказанных показателей качества, однако значительного ухудшения качества работы \fixme{модели} на валидационной выборке с течением времени не наблюдалось (см.~Рис.~\ref{fig:val-losses}; можно заметить небольшой рост значений функции потерь у \fixme{модели} на основе EfficientNetV2"~S, что свидетельствует о некотором переобучении \fixme{модели}), поэтому в качестве финального состояния принималось значение весов \fixme{модели} в конце последней эпохи.

\begin{figure}[ht]
	\centerfloat{
		\hfill
		\subcaptionbox[List-of-Figures entry]{ResNet"~18}{%
			\includegraphics[width=0.48\textwidth]{val_loss_ord-clahe2-ru.png}}
		\hfill
		\subcaptionbox{EfficientNetV2"~S}{%
			\includegraphics[width=0.48\textwidth]{val_loss_eff-ru.png}}
		\hfill
	}
	\caption{Примеры графиков зависимости функции потерь от количества эпох на валидационной выборке в задаче анализа жёсткости снимков грудной клетки}
	\label{fig:val-losses}
\end{figure}

Итоговые значения показателей качества, полученные на тестовой выборке, представлены в Табл.~\ref{tab:hardness-metrics-test}. \fixme{Модели} для решения задачи порядковой регрессии содержат «ord» в своём названии, \fixme{модель} для решения задачи классификации "--- <<clf>>. \fixme{Модель} <<ord"~eff>>, была создана на основе EfficientNetV2"~S, а остальные \fixme{модели} "--- на основе ResNet"~18. В столбце <<Эквализация гистограммы>> стоит прочерк, если эквализация не применялась; <<глобальная>> в случае использования глобальной эквализации; размер окна локальной эквализации гистограммы в виде доли от размера всего изображения в случае использования локальной эквализации.

На основании сбалансированной точности и сбалансированной MAE лучшей \fixme{моделью} оказалась \fixme{модель} порядковой регрессии с локальной эквализацией гистограммы на этапе предобработки снимков с окном, длина стороны которого была равна $\frac{1}{2}$ длины стороны изображения. Далее для краткости она будет обозначена <<ord"~clahe2>>. Дальнейший анализ изображений проводится с использованием этой \fixme{модели}.

\begin{table} [htbp]%
	\centering
	\caption{Значения показателей качества работы алгоритмов определения жёсткости на тестовой выборке набора SakhaTB}%
	\label{tab:hardness-metrics-test}% label всегда желательно идти после caption
	\renewcommand{\arraystretch}{1.5}%% Увеличение расстояния между рядами, для улучшения восприятия.
	\begin{SingleSpace}
		\begin{tabular}{@{}@{\extracolsep{20pt}}lccc@{}} %Вертикальные полосы не используются принципиально, как и лишние горизонтальные (допускается по ГОСТ 2.105 пункт 4.4.5) % @{} позволяет прижиматься к краям
			\toprule     %%% верхняя линейка
			\fixme{Модель} & Эквализация гистограммы & BalAcc & mMAE \\
			\midrule %%% тонкий разделитель. Отделяет названия столбцов. Обязателен по ГОСТ 2.105 пункт 4.4.5
			clf	& - & 0.563 & 0.452 \\
			ord"~eff & - & 0.623 & 0.399 \\
			ord & - & 0.609 & 0.452 \\
			ord"~glob & глобальная & 0.609 & 0.452 \\
			\textbf{ord"~clahe2} & \textbf{$\frac{1}{2}$ стороны изображения} & \textbf{0.636} & \textbf{0.398} \\
			ord"~clahe4 & $\frac{1}{4}$ стороны изображения	& 0.610 & 0.449 \\
			ord"~clahe8 & $\frac{1}{8}$ стороны изображения	& 0.593 & 0.468 \\
			ord"~clahe16 & $\frac{1}{16}$ стороны изображения & 0.600 & 0.468 \\
			\bottomrule %%% нижняя линейка
		\end{tabular}%
	\end{SingleSpace}
\end{table}

На Рис.~\ref{fig:hardness-class-separation} представлена зависимость вероятностей классов <<Жёсткий>>~(Hard) и <<Мягкий>>~(Soft), предсказанных \fixme{моделью} <<clf>> для решения задачи простой классификации, а также безразмерной величины жёсткости, предсказанной \fixme{моделью} <<ord"~clahe2>> для задачи порядковой регрессии, и истинного значения уровня жёсткости объектов тестовой выборки. Можно заметить, что разделение классов далеко от идеального.

\begin{figure}[ht]
	\centerfloat{
		\hfill
		\subcaptionbox[List-of-Figures entry]{<<clf>> (простая классификация)}{%
			\includegraphics[width=0.45\textwidth]{clf_separation.png}}
		\hfill
		\subcaptionbox{<<ord"~clahe2>> (порядковая регрессия)}{%
			\includegraphics[width=0.45\textwidth]{ord_separation.png}}
		\hfill
	}
	\caption{Зависимость предсказаний \fixme{моделей} в зависимости от истинного класса объекта для набора SakhaTB}
	\label{fig:hardness-class-separation}
\end{figure}

Как выяснилось после проверки ошибочно классифицированных снимков, причиной тому послужила шумность использованного набора данных из-за неоднозначности и недостаточной определённости критериев разметки. Подтверждением этого предположения также служит близость значений меры качества на обучающей выборке к таковым на тестовой выборке: сбалансированная точность около 0.70 и 0.67 у \fixme{моделей} <<clf>> и <<ord"~clahe2>> соответственно.

Рассмотрение задачи определения жёсткости рентгеновского снимка как задачи порядковой регрессии позволяет, используя полученную для её решения \fixme{модель}, с некоторой точностью ранжировать изображения по жёсткости на основе внутреннего показателя жёсткости нейронной сети. В качестве меры качества ранжирования был выбран коэффициент ранговой корреляции Спирмена~\cite{zwillinger1999crc}, так как он позволяет обнаруживать в том числе нелинейные зависимости рассматриваемых величин. Замеры делались относительно указанных врачом количества чётко видимых на снимке позвонков и относительно классов жёсткости снимков. Полученные значения приведены в Табл.~\ref{tab:hardness-spearman-test}.

\begin{table} [htbp]%
	\centering
	\caption{Значения показателя качества ранжирования тестовой выборки набора SakhaTB алгоритмом определения жёсткости}%
	\label{tab:hardness-spearman-test}% label всегда желательно идти после caption
	\renewcommand{\arraystretch}{1.5}%% Увеличение расстояния между рядами, для улучшения восприятия.
	\begin{SingleSpace}
		\begin{tabular}{@{}@{\extracolsep{20pt}}lcc@{}} %Вертикальные полосы не используются принципиально, как и лишние горизонтальные (допускается по ГОСТ 2.105 пункт 4.4.5) % @{} позволяет прижиматься к краям
			\toprule     %%% верхняя линейка
			\fixme{Модель} & Spearman (позвонки) & Spearman (жёсткость) \\
			\midrule %%% тонкий разделитель. Отделяет названия столбцов. Обязателен по ГОСТ 2.105 пункт 4.4.5
			ord"~eff & 0.564 & 0.457 \\
			ord & 0.576 & 0.497 \\
			ord"~glob & 0.599 & 0.514 \\
			\textbf{ord"~clahe2} & \textbf{0.606} & \textbf{0.534} \\
			ord"~clahe4 & 0.602 & 0.519 \\
			ord"~clahe8 & 0.588 & 0.498 \\
			ord"~clahe16 & 0.596 & 0.507 \\
			\bottomrule %%% нижняя линейка
		\end{tabular}%
	\end{SingleSpace}
\end{table}

Была проведена оценка качества работы \fixme{модели} <<ord"~clahe2>> на наборе MC"~SZ. Результаты сравнения качества работы алгоритма на наборе MC"~SZ и на тестовой выборке набора SakhaTB приведены в Табл.~\ref{tab:hardness-metrics-mc-sz}.

\begin{table} [htbp]%
	\centering
	\caption{Значения показателя качества ранжирования тестовой выборки набора SakhaTB алгоритмом определения жёсткости}%
	\label{tab:hardness-metrics-mc-sz}% label всегда желательно идти после caption
	\renewcommand{\arraystretch}{1.5}%% Увеличение расстояния между рядами, для улучшения восприятия.
	\begin{SingleSpace}
		\begin{tabulary}{0.9\textwidth}{@{}@{\extracolsep{20pt}}lCCCC@{}} %Вертикальные полосы не используются принципиально, как и лишние горизонтальные (допускается по ГОСТ 2.105 пункт 4.4.5) % @{} позволяет прижиматься к краям
			\toprule     %%% верхняя линейка
			Набор данных & BalAcc & mMAE & Spearman \mbox{(позвонки)} & Spearman \mbox{(жёсткость)} \\
			\midrule %%% тонкий разделитель. Отделяет названия столбцов. Обязателен по ГОСТ 2.105 пункт 4.4.5
			Набор SakhaTB & 0.636 & 0.398 & 0.606 & 0.534 \\
			MC"~SZ & 0.546 & 0.565 & 0.325 & 0.203 \\
			\bottomrule %%% нижняя линейка
		\end{tabulary}%
	\end{SingleSpace}
\end{table}

\section{Использование результатов оценки качества рентгеновских снимков при нейросетевой диагностике туберкулёза лёгких}

Результаты оценки жёсткости рентгенограмм могут быть использованы, например, для предварительной фильтрации обучающей выборки и входных данных нейросетевого алгоритма компьютерной диагностики туберкулёза лёгких. Как показано в работе~\cite{dovganich2022automatic}, такая фильтрация, а также создание отдельных \fixme{моделей} для разных уровней жёсткости снимков позволяют повысить робастность и точность метода компьютерной диагностики. В этом разделе демонстрируется влияние предварительной фильтрации обучающей и валидационной выборок на качество работы алгоритма классификации в задаче диагностики туберкулёза лёгких по рентгеновским снимкам грудной клетки.

\subsection{Формирование набора данных} \label{subsubsec:dataset-tbx}

Так как в экспериментах этапу обучения \fixme{модели} в задаче диагностики туберкулёза лёгких предшествует этап фильтрации снимков на основании уровня жёсткости, то в целях поддержания размера обучающей и валидационной выборок после прореживания на достаточном уровне, а также для создания более разнородных выборок, как и в работе ~\cite{dovganich2022automatic}, для создания нейросетевого метода диагностики и замера влияния фильтрации на качество диагностики использовался набор рентгенограмм, состоящий из двух частей: уже упомянутый в п.~\ref{subsubsec:dataset-mc-sz} составной набор MC"~SZ и ещё один из часто применяемых при разработке методов компьютерной диагностики туберкулёза лёгких общедоступных наборов рентгенограмм~\cite{singh2022evolution, santosh2022advances} "--- TBX11K~\cite{liu2020rethinking}.

%\subsubsection{Набор TBX11K} \label{subsubsec:dataset-tbx} % перенёс метку в родительскую \subsection

Набор TBX11K подготовлен в Нанькайском университете г.~Тяньцзин в КНР. Он состоит из 11200 рентгеновских снимков грудной клетки в оттенках серого с глубиной цвета 8 бит в формате PNG с разрешением 512x512 пикселей. Он изначально поделён на обучающую, валидационную и тестовую выборки, при этом последняя не имеет разметки, поэтому из всего набора только для 8400 изображений доступны метки принадлежности к одному из 3 классов (3800 здоровых и 800 больных туберкулёзом лёгких пациентов, а также больные иными заболеваниями пациенты) и границы поражённых областей лёгких. В рамках исследования использовались только снимки здоровых и больных туберкулёзом лёгких пациентов, а остальные были исключены из рассмотрения. Примеры изображений из набора представлены на Рис.~\ref{fig:samples-tbx}.

\begin{figure}[ht]
	\centerfloat{
%		\hfill
		\subcaptionbox[List-of-Figures entry]{Здоровые пациенты}{%
			\includegraphics[height=0.25\textheight]{tbx11k_normal.png}}
%		\hfill
		\vfill
%		\hfill
		\subcaptionbox{Пациенты, больные туберкулёзом лёгких}{%
			\includegraphics[height=0.25\textheight]{tbx11k_tb.png}}
%		\hfill
		\vfill
	}
	\caption{Примеры изображений из набора TBX11K}
	\label{fig:samples-tbx}
\end{figure}

Рентгенограммам здоровых пациентов из получившегося объединённого набора были поставлены в соответствие метки класса NORMAL, а больных туберкулёзом лёгких пациентов "--- класса TB. Итоговое число снимков каждого класса в объединённом наборе, а также в его отдельных частях приведено в Табл.~\ref{tab:hardness-dataset-size}.

\begin{table} [htbp]%
	\centering
	\caption{Размеры использованных наборов данных}%
	\label{tab:hardness-dataset-size}% label всегда желательно идти после caption
	\renewcommand{\arraystretch}{1.5}%% Увеличение расстояния между рядами, для улучшения восприятия.
	\begin{SingleSpace}
		\begin{tabulary}{\textwidth}{@{}@{\extracolsep{10pt}}lCCC@{}} %Вертикальные полосы не используются принципиально, как и лишние горизонтальные (допускается по ГОСТ 2.105 пункт 4.4.5) % @{} позволяет прижиматься к краям
			\toprule     %%% верхняя линейка
			Название набора & Число снимков класса NORMAL & Число снимков класса TB & Общее число снимков \\
			\midrule %%% тонкий разделитель. Отделяет названия столбцов. Обязателен по ГОСТ 2.105 пункт 4.4.5
			Montgomery & 80 & 58 & 138 \\
			Shenzhen & 326 & 336 & 662 \\
			TBX11K & 3800 & 800 & 4600 \\
			\midrule
			Всего & 4206 & 1194 & 5400 \\
			\bottomrule %%% нижняя линейка
		\end{tabulary}%
	\end{SingleSpace}
\end{table}

\subsection{Метод компьютерной диагностики туберкулёза лёгких}\label{subsec:tbx-diagnostics}

Использованный в этом разделе для экспериментов алгоритм диагностики туберкулёза лёгких выглядел следующим образом:
\begin{enumerate}[beginpenalty=10000]
	\item поступивший на вход рентгеновский снимок грудной клетки подвергается предобработке, совпадающей с описанной в п.~\ref{subsec:tb-hardness-preprocessing} за исключением того, что эквализация гистограммы не проводилась;
	\item при помощи нейросетевой \fixme{модели} ему ставятся в соответствие 2 вещественных числа из отрезка $\left[ 0, 1 \right]$, которые выражают предсказанные веса классов NORMAL и TB (сумма весов для каждого изображения равна 1);
	\item класс, чей вес больше, принимается за выходное значение алгоритма.
\end{enumerate}

В качестве основы для нейросетевой \fixme{модели} использовалась сеть EfficientNetV2"~S, последний слой которой был заменён на полносвязный слой с 2 выходами. Процедуры деления набора данных на подвыборки и обучения \fixme{модели} со взвешиванием классов для балансировки, а также начальные состояния весов сети совпадали с описанными в п.~\ref{subsec:tb-hardness-experiments}. В роли функции потерь выступала кросс"=энтропия, а в роли меры качества "--- сбалансированная точность.

%В качестве основ для нейронных сетей использовались EfficientNetV2"~S и ResNet"~18, последний слой которых заменялся на полносвязный с 2 выходами. Процедуры деления набора данных на подвыборки и обучения модели со взвешиванием классов для балансировки, а также начальные состояния весов и этапы предобработки были аналогичны таковым из метода в предыдущем разделе, однако эквализация гистограммы не проводилась. В роли функции потерь выступала кросс"=энтропия, а в роли меры качества "--- сбалансированная точность.
%
%При решении этой задачи из двух моделей лучшее качество классификации до прореживания показала модель на основе EfficientNetV2"~S, и именно она участвует в сравнении. Вероятно, это обусловлено значительно возросшим объёмом и качеством выборки. 

\subsection{Эксперименты и результаты}

Рентгенограммы из наборов MC"~SZ и TBX11K были обработаны разработанным алгоритмом оценки жёсткости. Распределение снимков из тестовой выборки сформированного в п.~\ref{subsubsec:dataset-yak-hardness} набора и из этих двух наборов по безразмерному показателю жёсткости, предсказанному \fixme{моделью} <<ord"~clahe2>>, представлен на Рис.~\ref{fig:ordinal-hist-total}. Однако следует осторожно относиться к разделению снимков из наборов MS"~SZ и TBX11K на классы жёсткости, так как истинные пороги разделения классов для этих данных могут сильно отличаться от полученных \fixme{моделью} на этапе обучения в п.~\ref{subsec:tb-hardness-experiments} из-за их визуального отличия от изображений её обучающей выборки.

\begin{figure}[ht]
	\centerfloat{
		\includegraphics[height=0.3\textheight]{total_ordinal_hist_ru.png}
	}
	\caption{Гистограммы распределения снимков трёх использованных наборов изображений по предсказанному \fixme{моделью} <<ord"~clahe2>> показателю жёсткости}
	\label{fig:ordinal-hist-total}
\end{figure}

Отдельные гистограммы для классов NORMAL и TB для наборов MC"~SZ и TBX11K представлены на Рис.~\ref{fig:ordinal-hist-datasetwise}. Небольшие различия гистограмм классов совпадают с визуальной разницей между снимками этих классов: класс TB в обоих из них содержит больше мягких снимков, а классы NORMAL "--- больше жёстких (см.~Рис.~\ref{fig:samples-mc-sz}~и~\ref{fig:samples-tbx}).

\begin{figure}[ht]
	\centerfloat{
		\subcaptionbox[List-of-Figures entry]{MC"~SZ}{%
			\includegraphics[height=0.25\textheight]{mcsz_ordinal_hist_ru_1.png}}
		
		\subcaptionbox{TBX11K}{%
			\includegraphics[height=0.25\textheight]{tbx11k_ordinal_hist_ru_1.png}}
	}
	\caption{Гистограммы распределения снимков каждого класса для использованных в задаче диагностики туберкулёза лёгких наборов изображений MC"~SZ и TBX11K по предсказанному \fixme{моделью} <<ord"~clahe2>> показателю жёсткости}
	\label{fig:ordinal-hist-datasetwise}
\end{figure}

На основании предсказанных величин жёсткости было проведено удаление из объединённого набора MC"~SZ"~TBX11K одинаковой доли самых жёстких и самых мягких снимков (то есть с обеих сторон гистограммы распределения величин жёсткости). После этого качество обученного на прореженной обучающей выборке метода диагностики замерялось на прореженной тестовой выборке и сравнивалось с качеством работы на такой же тестовой выборке \fixme{модели} диагностики, обученной на непрореженной обучающей выборке.

Исходя из визуального различия составных частей используемого набора данных и относительно малого размера набора MC"~SZ, было принято решение прореживать не объединённый набор изображений, а каждую из двух его частей отдельно, чтобы изменение их пропорций в итоговой выборке не повлияло на качество работы. Такой выборочный подход обусловлен необходимостью более точного контрастирования снимков, полученных на разных аппаратах и в разных условиях, и приведения их к как можно более близкому внешнему виду перед определением жёсткости.

Помимо прореживания выборок с обеих сторон гистограммы уровня жёсткости был рассмотрен и случай отбрасывания только самых жёстких снимков, так как на мягких снимках ещё могут сохраняться некоторые детали лёгочной ткани, в том время как на жёстких они могут быть полностью утеряны.

Значения сбалансированной точности, полученные в результате экспериментов представлены в Табл.~\ref{tab:hardness-filtering-balacc}. Из неё видно, что изменение качества зависит от степени прореживания изображений, однако остаётся стабильно положительным. Сравнение более распространённых в медицине показателей чувствительности и специфичности для класса TB приведено в Табл.~\ref{tab:hardness-filtering-sens-spec}.

\begin{table} [htbp]%
	\centering
	\caption{Сравнение качества классификации \fixme{моделей}, обученных на полном и прореженном наборе (сбалансированная точность)}%
	\label{tab:hardness-filtering-balacc}% label всегда желательно идти после caption
	\renewcommand{\arraystretch}{1.5}%% Увеличение расстояния между рядами, для улучшения восприятия.
	\begin{SingleSpace}
		\begin{tabulary}{\textwidth}{@{}@{\extracolsep{10pt}}cCCCCCC@{}} %Вертикальные полосы не используются принципиально, как и лишние горизонтальные (допускается по ГОСТ 2.105 пункт 4.4.5) % @{} позволяет прижиматься к краям
			\toprule     %%% верхняя линейка
			& \multicolumn{3}{@{}c@{}}{\makecell{Удаление жёстких и \\ мягких изображений}} & \multicolumn{3}{@{}c@{}}{\makecell{Удаление только \\ жёстких изображений}} \\
			\cmidrule(r){2-4}\cmidrule(l){5-7}
			Доля удалённых & 5\% & 10\% & 15\% & 5\% & 10\% & 15\% \\
			\midrule %%% тонкий разделитель. Отделяет названия столбцов. Обязателен по ГОСТ 2.105 пункт 4.4.5
			До & 0.958 & 0.951 & 0.951 & 0.962 & 0.961 & 0.965 \\
			После & 0.961 & 0.962 & 0.953 & 0.968 & 0.966 & 0.975 \\
			\bottomrule %%% нижняя линейка
		\end{tabulary}%
	\end{SingleSpace}
\end{table}

\begin{table} [htbp]%
	\centering
	\caption{Сравнение качества классификации \fixme{моделей}, обученных на полном и прореженном наборе (чувствительность~/~специфичность)}%
	\label{tab:hardness-filtering-sens-spec}% label всегда желательно идти после caption
	\renewcommand{\arraystretch}{1.5}%% Увеличение расстояния между рядами, для улучшения восприятия.
	\begin{SingleSpace}
		\begin{tabulary}{\textwidth}{@{}@{\extracolsep{10pt}}cCCCCCC@{}} %Вертикальные полосы не используются принципиально, как и лишние горизонтальные (допускается по ГОСТ 2.105 пункт 4.4.5) % @{} позволяет прижиматься к краям
			\toprule     %%% верхняя линейка
			& \multicolumn{3}{@{}c@{}}{\makecell{Удаление жёстких и \\ мягких изображений}} & \multicolumn{3}{@{}c@{}}{\makecell{Удаление только \\ жёстких изображений}} \\
			\cmidrule(r){2-4}\cmidrule(l){5-7}
			Доля удалённых & 5\% & 10\% & 15\% & 5\% & 10\% & 15\% \\
			\midrule %%% тонкий разделитель. Отделяет названия столбцов. Обязателен по ГОСТ 2.105 пункт 4.4.5
			До & 0.923/ 0.994 & 0.909/ 0.994 & 0.908/ 0.995 & 0.930/ 0.994 & 0.927/ 0.995 & 0.934/ 0.996 \\
			После & 0.933/ 0.990 & 0.933/ 0.991 & 0.915/ 0.990 & 0.943/ 0.994 & 0.941/ 0.992 & 0.958/ 0.993 \\
			\bottomrule %%% нижняя линейка
		\end{tabulary}%
	\end{SingleSpace}
\end{table}

\section{Выводы} 

В данной главе была продемонстрирована возможность использования нейросетевых методов для определения уровня жёсткости рентгеновских снимков грудной клетки. Хотя из-за особенности имеющихся данных не удалось достичь высоких показателей качества, полученные результаты заметно превышают возможности случайного выбора ответа: сбалансированная точность составила 0.636, а коэффициенты корреляции Спирмена "--- 0.606 и 0.534. Однако качество работы полученного алгоритма ощутимо снижается в случае применения его к данным из других источников: для набора MC"~SZ сбалансированная точность упала до 0.546, а значения показателя качества ранжирования объектов снизились примерно вдвое.

Однако даже такая несовершенная \fixme{модель} определения жёсткости позволяет увеличить качество работы алгоритма диагностики туберкулёза лёгких при условии предварительной фильтрации изображений перед обучением классификатора и получением предсказаний. Снижение разброса жёсткости данных и увеличение их однородности заметно повысило точность обнаружения больных пациентов при сохранении или малом снижении специфичности: наибольший абсолютный и относительный прирост чувствительности алгоритма для класса TB наблюдался при удалении 10\% самых жёстких и 10\% самых мягких снимков (с 0.909 до 0.933) и при удалении 15\% самых жёстких снимков (с 0.934 до 0.958). Во втором случае также было достигнуто наибольшее значение чувствительности: 0.958.


\FloatBarrier 
