%TODO: Смотри в самый низ - там кусок про создание Sakha-TB

\chapter{Программный комплекс реализации алгоритмов анализа биомедицинских изображений}\label{ch:ch4}

Данная глава посвящена программной реализации разработанных ранее алгоритмов и методов. Так как рассмотренные в работе задачи относятся к разным областям медицины и биологии и решают задачи разного типа, то разработанные методы и алгоритмы были реализованы в виде независимых программных модулей.

Программный комплекс, разработанный в рамках данной главы, содержит в себе четыре отдельных модуля:

\begin{enumerate}[beginpenalty=10000]
	\item модуль повышения разрешения изображений мигающей флуоресцентной микроскопии;
	
	\item модуль повышения резкости медицинских изображений методом деформации пиксельной сетки;
	
	\item модуль анализа и обработки рентгенограмм грудной клетки;
	
	\item набор рентгеновских изображений грудной клетки для компьютерной диагностики туберкулёза.
\end{enumerate}

Ниже представлено детальное описание каждого из модулей.
 
\section{Программный модуль повышения разрешения изображений мигающей флуоресцентной микроскопии}
%TODO

\section{Программный повышения резкости медицинских изображений методом деформации пиксельной сетки}
%TODO

\section{Программный модуль анализа и обработки рентгенограмм грудной клетки}
%TODO

\section{Набор рентгеновских изображений грудной клетки для компьютерной диагностики туберкулёза лёгких} \label{sec:sakha-tb}
%TODO

\subsubsection{Наборы рентгенограмм грудной клетки DA и DB}

Наборы DA и DB~\cite{chauhan2014role} были собраны в Национальном институте туберкулёза и респираторных заболеваний в Нью-Дели в Индии. Набор DA содержит по 78 изображений здоровых и больных туберкулёзом лёгких пациентов в оттенках серого в формате JPG с разрешением 1024x1024 или 2320x2828 пикселей и глубиной цвета 8 бит. Набор DB содержит изображения в оттенках серого в формате DICOM с разрешением 3008x3008 пикселей и глубиной цвета 16 бит, и в рамках данного исследования был преобразован в формат PNG с глубиной цвета 8 бит. Некоторые снимки больных туберкулёзом лёгких из набора DB отсутствуют в его репозитории на сайте SourceForge, поэтому итоговое число снимков в нём составило 75 рентгенограмм здоровых и 47 рентгенограмм больных туберкулёзом лёгких пациентов. Эти два набора в силу их родства и малого размера по отдельности в рамках данной работы были объединены вместе. Примеры изображений из набора представлены на Рис.~\ref{fig:samples-dadb}.

%TODO Рисунок "Примеры DA и DB"

\subsubsection{Набор рентгенограмм грудной клетки Sakha"~TB}

Данный набор снимков, сделанных во фронтальной проекции, собран в результате сотрудничества с несколькими медицинскими учреждениями Республики Саха (Якутия); ниже он обозначен Sakha"~TB. Процедура формирования набора изображений из имеющихся коллекций рентгенограмм включала в себя:

\begin{itemize}
	\item удаление снимков с боковой проекцией;
	\item контроль правильности установленных значений <<Photometric Interpretation>> в файлах DICOM;
	\item удаление повторяющихся сессий;
	\item удаление пациентов с несколькими сессиями и различиями в диагнозе между сессиями;
	\item удаление несовершеннолетних пациентов;
	\item а также некоторую балансировку соотношения полов и распределений возрастов и диагнозов.
\end{itemize}

В результате было отобрано 400 изображений здоровых и 400 изображений больных туберкулёзом лёгких пациентов, где каждому пациенту соответствует только 1 снимок. Для получения рентгенограмм использовались стационарные и переносные комплексы оборудования. В основном разрешение изображений примерно равно 3000x3000 пикселей, но часть снимков имеет меньший размер вплоть до около 2000x2000 пикселей. Большинство изображений имеет глубину цвета 16 бит, у остальных снимков она равна 8 битам. Перед дальнейшим использованием изображения подверглись обработке как в п.~\ref{subsec:dataset-yak-1}.

Врачебная диагностика осуществлялась посредством независимого двойного чтения снимков с подтверждением диагноза <<туберкулёз>> специалистами НПЦ~<<Фтизиатрия>> им. Е.Н.~Андреева на основании клинико-лабораторных и микробиологических данных.

Примеры рентгенограмм из сформированного набора изображений представлены на Рис.~\ref{fig:samples-yak-2}. Сравнение распределений снимков по диагнозу, полу и возрасту полученного набора и набора MC+SZ (для наборов TBX11K, DA и DB такие данные недоступны) изображены на Рис.~\ref{fig:sex-distr}"~\ref{fig:age-distr-tb}.

%TODO Рисунки "Примеры Sakha-TB" и диаграммы распределений


\clearpage