\chapter{Повышение резкости медицинских изображений методом деформации пиксельной сетки}\label{ch:ch2}

На практике часто возникает необходимость повышения резкости изображений по разным причинам: из-за физических ограничений оптических систем и сенсоров камер или из-за сложных условий получения изображений. 
%В то же время в медицине при обработке изображений важно не терять присутствующие на них детали и не добавлять не существовавшие элементы или артефакты.
Повышение резкости изображений "--- это сложная некорректно поставленная задача восстановления резкого изображения из заданного размытого и зашумлённого изображения. Разработаны регуляризирующие алгоритмы для реконструкции исходного изображения при условии, что ядро размытия и шум известны с достаточной точностью, а также для реконструкции исходного изображения вместе с одновременной оценкой ядра размытия при отсутствии информации о нём (см.~\cite{wang2014recent}). В последнем случае для эффективной работы метода необходима информация о классе <<естественных>> изображений. Обычно такие алгоритмы реконструкции имеют параметры, которые определяют баланс между резкостью результата и вероятностью усиления шума или возникновения эффекта ложного оконтуривания. Кроме того, создан ряд методов глубокого обучения для непосредственного восстановления резких изображений (см.~\cite{SU202246}), а также для устранения артефактов шума после восстановления изображения классическими алгоритмами и для оценки распределения <<естественных>> изображений, которое может применяться в классических регуляризирующих алгоритмах устранения размытия (см.~\cite{Zhang2022}).

Несмотря на это, сложностью при решении задачи устранения размытия остаётся то, что реальные изображения, помимо размытия, подвержены квантованию, артефактам дебайеризации и другим искажениям. Кроме того, модельный шум может не в полной мере соответствовать шуму на реальных изображениях, а разработанные для одних видов изображений методы могут показывать плохое качество обработки изображений из другой области.

Особо остро стоит вопрос обработки медицинских изображений из-за высокой цены ошибки в этой сфере деятельности. Повышение резкости медицинских изображений положительно влияет на их наглядность, но в то же время артефакты обработки могут снижать качество диагностики (см., например,~\cite{Nakajima2023}~и~\cite{CLARK2018927}). Так как зачастую сложно достичь компромисса между резкостью результата и уровнем артефактов, то сохраняется актуальность создания алгоритмов повышения резкости медицинских изображений, которые при этом не снижают качество удалённых от контуров областей изображения и минимизируют число артефактов на результирующем изображении.

В отличие от наиболее распространённых методов повышения резкости (обращение свёртки, нерезкое маскирование), деформационный метод, представленный в~\cite{krylov2014gridwarping} и затем доработанный в~\cite{nasonova2014deblurred, gusev2016parallel}, не меняет значения интенсивности пикселей напрямую, а сдвигает сами пиксели в окрестности контуров изображений ближе к этим контурам~(см.~Рис~\ref{fig:warping-warping-idea} и~\ref{fig:warp-idea}). Этот подход изменяет изображение только в окрестности контуров и не имеет таких недостатков, свойственных некоторым другим методам, как усиление шума и возникновение эффекта ложного оконтуривания (эффекта Гиббса).

\begin{figure}[ht]
	\centerfloat{
		\hfill
		\subcaptionbox[List-of-Figures entry]{Профиль контура}{%
			\includegraphics[width=0.3\textwidth]{warping-1-4b.png}}
		\hfill
		\subcaptionbox{Классический метод}{%
			\includegraphics[width=0.3\textwidth]{warping-1-4a.png}}
		\hfill
		\subcaptionbox{Деформационный метод}{%
			\includegraphics[width=0.3\textwidth]{warping-1-4c.png}}
		\hfill
	}
	\caption{Принцип работы классических методов и деформационного метода повышения резкости изображений в одномерном случае}
	\label{fig:warping-warping-idea}
\end{figure}

\begin{figure}[ht]
	\centerfloat{
		\includegraphics[height=0.25\textheight]{warp-idea.png}
	}
	\caption{Принцип работы деформационного метода повышения резкости изображений в двумерном случае: смещение пикселей окрестности контура (чёрные пиксели) по направлению к контуру (белые пиксели)}
	\label{fig:warp-idea}
\end{figure}

В данной главе рассматривается деформационный алгоритм повышения резкости изображения~\cite{nasonova2014deblurred, gusev2016parallel} и набор из 24 тестовых изображений из базы TID2013~\cite{ponomarenko2015image}, размытых с разной интенсивностью с использованием трёх моделей реалистичных ядер оптического размытия. Целью является нахождение оптимальной функции смещения для каждой модели ядра размытия.% отдельно и для всех моделей сразу.

\section{Модели реалистичных ядер размытия изображений}

%Модель размытия изображения, рассматриваемая в этой главе, совпадает с моделью, описанной в п.~\ref{sec:image-blur-model}.

Рассматривается следующая математическая модель размытия изображения:

\begin{equation*}
	%	I^\prime \left( x, y \right)=\left(K\ast I+N\right)\left(x,y\right)=\int_{-\infty}^{\infty}\int_{-\infty}^{\infty}{K\left(s,t\right)\ I\left(x-s,y-t\right)\ ds\ dt}+N(x,y),
	I^\prime \left(x, y\right) = \left(I \ast f\right) \left(x, y\right) = \int_{-\infty}^{\infty}\int_{-\infty}^{\infty}{I\left(s,t\right)\ f\left(x-s,y-t\right)\ ds\ dt},
\end{equation*}

\noindent где $I$ "--- исходное резкое изображение, $I^\prime$ "--- размытое изображение, $f$ "--- ядро размытия. В дискретном случае она принимает вид:

\begin{equation*}
	I^\prime_{i,j} = \left(I \ast f\right)_{i,j} = \sum_{k=-\infty}^{\infty} \sum_{l=-\infty}^{\infty}{I_{k,l}\ f_{i-k,j-l}}.
\end{equation*}

В процессе исследования рассматривались три типа ядер оптического размытия, возникающих на практике: ядро, определяемое функцией Гаусса; и два ядра, которые часто возникают в реальных изображения, полученных с помощью фотокамер в случае нахождения объекта вне плоскости фокусировки.

Гауссово ядро выбрано по той причине, что оно соответствует одной из наиболее распространённых моделей размытия изображения, а также потому, что такое ядро (или очень близкое к нему) можно встретить в изображениях, полученных с помощью ряда оптических систем.

Согласно модели Зейделя, существует пять основных видов оптических аберраций~\cite{simpkins2014parameterized} (примеры этих аберраций представлены на Рис.~\ref{fig:warping-aberrations}):
\begin{enumerate}[beginpenalty=10000]
	\item сферическая аберрация;
	\item кома;
	\item астигматизм;
	\item кривизна поля изображения;
	\item дисторсия.
\end{enumerate}

\noindent Ядра, отвечающие сферическим аберрациям, также можно аппроксимировать гауссовым ядром~\cite{Miks:19, simpkins2014parameterized, simpkins2011modeling}, что послужило дополнительным аргументом в пользу рассмотрения этого ядра.

%\begin{figure}[ht]
%	\centerfloat{
%		\hfill
%		\subcaptionbox[List-of-Figures entry]{Сферическая аберрация}{%
%			\includegraphics[width=0.45\textwidth]{warping-1-1a.png}}
%		\hfill
%		\subcaptionbox{Кома}{%
%			\includegraphics[width=0.45\textwidth]{warping-1-1b.png}}
%		\hfill
%		\vfill
%		\hfill
%		\subcaptionbox{Астигматизм}{%
%			\includegraphics[width=0.45\textwidth]{warping-1-1c.png}}
%		\hfill
%		\subcaptionbox{Две модели ядра размытия при астигматизме}{%
%			\includegraphics[width=0.45\textwidth]{warping-1-1d.png}}
%		\hfill
%		\vfill
%		\hfill
%		\subcaptionbox{Кривизна поля изображения}{%
%			\includegraphics[width=0.45\textwidth]{warping-1-1e.png}}
%		\hfill
%		\subcaptionbox{Дисторсия}{%
%			\includegraphics[width=0.45\textwidth]{warping-1-1f.png}}
%		\hfill
%		\vfill
%	}
%	\caption{Виды оптических аберраций~\cite{simpkins2011modeling}}
%	\label{fig:warping-aberrations}
%\end{figure}

\begin{figure}[ht]
	\centerfloat{
		\hfill
		\subcaptionbox[List-of-Figures entry]{Сферическая аберрация}{%
			\includegraphics[width=0.45\textwidth]{warping-1-1a.png}}
		\hfill
		\subcaptionbox{Кома}{%
			\includegraphics[width=0.45\textwidth]{warping-1-1b.png}}
		\hfill
		\vfill
		\hfill
		\subcaptionbox{Астигматизм}{%
			\includegraphics[width=0.45\textwidth]{warping-1-1c.png}}
		\hfill
		\subcaptionbox{Кривизна поля изображения}{%
			\includegraphics[width=0.45\textwidth]{warping-1-1e.png}}
		\hfill
		\vfill
%		\hfill
		\subcaptionbox{Дисторсия}{%
			\includegraphics[width=0.45\textwidth]{warping-1-1f.png}}
%		\hfill
%		\vfill
	}
	\caption{Виды оптических аберраций~\cite{simpkins2011modeling}}
	\label{fig:warping-aberrations}
\end{figure}

Помимо этих пяти аберраций также встречаются просто расфокусированные изображения, либо же расфокусированные участки изображений, полученных с использованием камер с малой глубиной резкости изображаемого пространства. В таком случае полученное изображение точки упрощённо имеет вид круга с приблизительно равномерной яркостью, однако также может иметь увеличение яркости в центре или на краях (что выглядит как круг с кольцом). Таким образом, ядро размытия часто имеет упрощённо вид круга или круга с ярким кольцом на границе. Конкретный вид зависит в том числе от положения объекта на изображении относительно области, находящейся  в фокусе. Примеры этого эффекта приведены на Рис.~\ref{fig:warping-defocus}.

\begin{figure}[ht]
	\centerfloat{
		\hfill
%		\subcaptionbox[List-of-Figures entry]{Объект за областью резкости}{%
%			\includegraphics[width=0.25\textwidth]{warping-1-2a.png}}
%		\hfill
		\subcaptionbox{Объект за областью резкости}{%
			\includegraphics[width=0.25\textwidth]{warping-1-2b_.png}}
		\hfill
		\subcaptionbox{Объект перед областью резкости}{%
			\includegraphics[width=0.25\textwidth]{warping-1-2c_.png}}
		\hfill
	}
	\caption{Примеры эффекта расфокусировки}
	\label{fig:warping-defocus}
\end{figure}

Ниже приведено описание использованных в ходе исследования моделей ядер размытия с параметрами $\sigma$ и $r$:
\begin{enumerate}[beginpenalty=10000]
	\item Гауссово ядро.
	Задаётся формулой
	$$f_\sigma\left(x,y\right) = \frac{1}{2\pi\sigma^2}\exp\left(-\frac{x^2+y^2}{2\sigma^2}\right).$$
%	Радиусом этого ядра будет считаться величина $3\sigma$.
%	Радиусом этого ядра будет считаться величина $1.5\sigma$.
	
	\item Ядро типа <<круг>>.
	Задаётся формулой $$f_r\left(x,y\right) = \begin{cases}
		1, & x^2 + y^2 \leq r^2, \\
		0, & \text{иначе}.
	\end{cases}$$
%	Радиус ядра "--- радиус круга $r$.
	
	\item Ядро типа <<круг с кольцом>>.
	Задаётся формулой $$f_r(x, y) = \begin{cases}
		0.25, & x^2 + y^2 \leq 0.75 r^2,\\
		1, & 0.75 r^2 < x^2 + y^2 \leq r^2,\\
		0, & \text{иначе}.
	\end{cases}$$
%	Радиус ядра "--- радиус круга $r$.
\end{enumerate}

Примеры этих ядер представлены на Рис.~\ref{fig:warping-psf-examples}. %Для каждой модели ядра был определён её <<радиус ядра размытия>>, равный $3\sigma$ для гауссова ядра и $r$ для двух оставшихся.

\begin{figure}[ht]
	\centerfloat{
		\hfill
		\subcaptionbox[List-of-Figures entry]{Гауссово ядро (Г)}{%
			\includegraphics[width=0.25\textwidth]{warping-1-3a.png}}
		\hfill
		\subcaptionbox{Ядро типа <<круг>> (К)}{%
			\includegraphics[width=0.25\textwidth]{warping-1-3b.png}}
		\hfill
		\subcaptionbox{Ядро типа <<круг с кольцом>> (КЦ)}{%
			\includegraphics[width=0.25\textwidth]{warping-1-3c.png}}
		\hfill
	}
	\caption{Рассматриваемые ядра размытия}
	\label{fig:warping-psf-examples}
\end{figure}

\section{Построение тестового набора изображений}

Для разработки метода повышения резкости использовались 24 базовых естественных изображения из набора изображений TID2013~\cite{ponomarenko2015image}, который является одним из стандартных наборов данных для тестирования методов обработки изображений. Изображения представлены на Рис.~\ref{fig:tid2013-db}.

\begin{figure}[ht]
	\centerfloat{
		\includegraphics[height=0.3\textheight]{tid2013-db.png}
	}
	\caption{Базовые изображения из набора TID2013}
	\label{fig:tid2013-db}
\end{figure}

При построении тестового набора к каждому изображению, переведённому в градации серого, применялся оператор свёртки с каждым из трёх рассматриваемых типов ядер размытия с разными значениями параметров размытия. Для ядер типа <<круг>> и <<круг с кольцом>> параметр $r$ изменялся в пределах от 1.5 до 5 пикселей с шагом 0.5 пикселя. Для гауссова ядра параметр $\sigma$ изменялся в пределах от 0.75 до 2.5 с шагом 0.25.
В результате для каждого типа размытия был сформирован набор из 192 изображений. Примеры изображений приведены на Рис.~\ref{fig:blurry-examples}.

\begin{figure}[ht]
	\centerfloat{
		\hfill
		\subcaptionbox[List-of-Figures entry]{Гауссово ядро, $\sigma=2.5$}{%
			\includegraphics[width=0.3\textwidth]{gauss24_5.png}}
		\hfill
		\subcaptionbox{Ядро типа <<круг>>, $r=2.5$}{%
			\includegraphics[width=0.3\textwidth]{circle24_5.png}}
		\hfill
		\subcaptionbox{Ядро типа <<круг с кольцом>>, $r=2.5$}{%
			\includegraphics[width=0.3\textwidth]{dcircle24_5.png}}
		\hfill
	}
	\caption{Фрагменты размытых изображений из сформированного набора данных}
	\label{fig:blurry-examples}
\end{figure}


\section{Деформационный метод повышения резкости изображений}

В процессе работы алгоритма:
\begin{enumerate}[beginpenalty=10000]
	\item на изображении обнаруживаются контуры алгоритмом Кэнни~\cite{canny1986computational},
	\item оценивается значение параметра размытия (например, методами, описанными в~\cite{nasonov2015edge}, \cite{isprs-archives-XLII-2-W12-161-2019}, \cite{6583957}, \cite{7589980} или~\cite{Lee_2019_CVPR}),
	\item производится вычисление векторов смещения пикселей в окрестности обнаруженных контуров по направлению к этим контурам,
	\item в соответствии с вычисленным векторным полем смещений деформируется пиксельная сетка,
	\item полученный результат проецируется на равномерную сетку выходного изображения с помощью интерполяции.
	%, описанной в~\cite{krylov2014gridwarping}.
\end{enumerate}

%\subsection{Вычисление значения параметра размытия}
%
%%TODO 

\subsection{Вычисление векторов смещений узлов пиксельной сетки}

Направление и величина смещения пикселей задаётся функцией смещения $d\left(x\right)$, $x_{new} \leftarrow x_{old}+d(x_{old})$. В одномерном случае, если центр контура находится в точке $x = 0$, вектора смещения определяются функцией близости $p\left(x\right)=1+d^{\,\prime}(x)$.

Тогда функция смещения может быть выражена через $p\left(x\right)$ с помощью уравнения:

\begin{equation*}
	d\left(x\right)=\int_{-\infty}^{x}\left(p\left(y\right)-1\right)dy.
\end{equation*}

Функция $p(x)$ выражает расстояние между соседними пикселями изображения: если её значение меньше 1, в точке $x$ уплотнение; если больше 1 "--- разрежение. Для недеформированных изображений $p(x)\equiv1$.

Чтобы взаимное расположение точек при смещении не менялось, для функции $d(x)$ должно быть выполнено $x_1<x_2 \Rightarrow x_1+d(x_1)\le x_2+d(x_2)$. Отсюда получаем ограничение $d^{\,\prime}\left(x\right)\geq-1$. Также деформироваться должна только сетка вблизи контуров, поэтому $d\left(x\right)\rightarrow0$ при  $\left|x\right|\rightarrow\infty$. Пример функции близости приведён на Рис.~\ref{fig:warping-proximity}.

\begin{figure}[ht]
	\centerfloat{
		\includegraphics[height=0.4\textheight]{warping-1-5.png}
	}
	\caption{Пример функции близости и результата смещения пикселей с её помощью}
	\label{fig:warping-proximity}
\end{figure}

В двумерном случае смещение задаётся векторным полем $\vec{d}(x,y)$, которое удовлетворяет уравнению $p\left(x,y\right)=1+div\,\vec{d}(x,y)$. Помимо этого, во избежание возникновения завихрений требуется ограничение $rot\, \vec{d}=0$, а так как $rot\, \nabla u \equiv 0$, то, если функция $p(x, y)$ известна, векторное поле смещений можно представить как $\vec{d}\left(x,y\right)=\nabla u\left(x,y\right)$, где функция $u\left(x,y\right)$ является решением уравнения

\begin{equation*}
	\left\{
		\begin{aligned}
			&\Delta u = p\left(x,y\right) - 1, \\
			&u\mid_\Gamma=0,
		\end{aligned}
	\right.
\end{equation*}

\noindent где $\Gamma$ "--- граница изображения.

Недостатком такого подхода является то, что контуры на изображении подвергаются смещению вместе с остальными пикселями, поэтому в работе~\cite{gusev2016parallel} предлагается для двумерного случая использовать функцию близости одной переменной для построения функции близости на плоскости с опорой на найденные контуры:

\begin{align*}
	&p\left(x,y\right) = \frac{\sum_{\left(x_e,y_e\right) \in N\left(x,y\right)}{w\left(x_e,y_e\right)p\left(x_n\right)}} {\sum_{\left(x_e,y_e\right) \in N\left(x,y\right)}{w\left(x_e,y_e\right)}}, \\
	&w\left(x_e,y_e\right) = G_{\widetilde{\sigma}}\left(x_t\right)\lvert \vec{g}\left(x_e,y_e\right) \rvert,
\end{align*}

\noindent где $N\left(x,y\right)$ "--- это множество точек контуров в окрестности т.~$\left(x, y\right)$; $x_n$ и $x_t$ "--- это проекции вектора $\left(x-x_e, y-y_e\right)$ на вектор градиента изображения $\vec{g}\left(x_e,y_e\right)$ и на нормаль к нему соответственно; а $G_{\widetilde{\sigma}}\left(x\right) = \frac{1}{\widetilde{\sigma} \sqrt{2\pi}} \exp\left(\frac{-x^2}{2\widetilde{\sigma}^2}\right)$, где $\widetilde{\sigma}$ "--- настраиваемый параметр, для которого в ходе выполнения исследования использовалось значение, равное $2.5$ параметрам размытия.

В этом случае итоговая функция смещения имеет следующий вид~\cite{gusev2016parallel}:

\begin{equation*}
	\vec{d}\left(x,y\right) = \frac{\sum_{\left(x_e,y_e\right) \in N\left(x,y\right)}{d\left(x_n\right) G_{\widetilde{\sigma}}\left(x_t\right) \vec{g}\left(x_e,y_e\right) }} {\sum_{\left(x_e,y_e\right) \in N\left(x,y\right)}{G_{\widetilde{\sigma}}\left(x_t\right)\lvert \vec{g}\left(x_e,y_e\right) \rvert}}.
\end{equation*}

Таким образом, можно перейти к работе с функцией смещения напрямую и проанализировать её влияние на результаты работы деформационного алгоритма повышения резкости.

\subsection{Интерполяция на равномерную сетку}

Интерполяция с деформированной сетки на исходную равномерную осуществляется следующим образом (см.~Рис.~\ref{fig:warping-interpolation} и~\cite{krylov2014gridwarping}). Значение интенсивности пикселя обработанного изображения $I_w(P)$ вычисляется как взвешенная сумма значений интенсивностей пикселей смещённой сетки в окрестности пикселя $P$ $N\left(P\right) = \left\{ Q: \lvert Q - P \rvert \leq \rho\ \right\}$:

\begin{equation*}
	I_w(P)=\frac{\sum_{Q \in N\left(P\right)} \frac{1}{\lvert P - Q \rvert} I(Q)}{\sum_{Q \in N\left(P\right)} \frac{1}{\lvert P - Q \rvert}}.
\end{equation*}

В рамках выполнения диссертационной работы значение $\rho$ было принято равным $1.5$.

\begin{figure}[ht]
	\centerfloat{
		\hfill
		\subcaptionbox[List-of-Figures entry]{Смещение пикселей}{%
			\includegraphics[width=0.3\textwidth]{int1.png}}
		\hfill
		\subcaptionbox{Наложение деформированной сетки на равномерную}{%
			\includegraphics[width=0.3\textwidth]{int2.png}}
		\hfill
		\subcaptionbox{Интерполяция}{%
			\includegraphics[width=0.3\textwidth]{int3.png}}
		\hfill
	}
	\caption{Фрагменты размытых изображений из сформированного набора данных}
	\label{fig:warping-interpolation}
\end{figure}


\section{Модели функции смещения}

В данной работе предлагаются две новые модели функции смещения:

\begin{enumerate}[beginpenalty=10000]	
	\item
	$
	\begin{aligned}
		d_2\left(x; a, b, c\right)=\left\{
		\begin{aligned}
			&\frac{c}{a}x,\ \left|x\right|<a,\\
			&c\frac{b-\left|x\right|}{b-a} \cdot sign\left(x\right),\ a\le\left|x\right|<b,\\
			&0,\ \left|x\right|\geq b;
		\end{aligned}
		\right.
	\end{aligned}
	$
	
	\item $d_1\left(x; a, c\right) = d_2\left(x; a, 1.5a, c\right)$.
\end{enumerate}

Также была рассмотрена и использована для оценки повышения эффективности алгоритма модель функции смещения, соответствующая функции близости, использованной в~\cite{nasonova2014deblurred, gusev2016parallel}:

$d_0\left(x; s\right)=s\sqrt\pi\left[erf\left(\frac{x}{2s}\right)-erf\left(\frac{x}{s}\right)\right]$, где $erf{\left(x\right)}=\frac{2}{\sqrt\pi}\int_{0}^{x}{e^{-t^2}dt}$.

При этом величина $x$ "--- это расстояние от точки до контура, делённое на параметр ядра размытия, что позволяет масштабировать функцию для разных уровней размытия без изменения самих параметров $a$, $b$, $c$ и $s$.

Пример функции $d_2(x; a, b, c)$ приведёт на Рис.~\ref{fig:warping-d2-example}.

\begin{figure}[ht]
	\centerfloat{
		\includegraphics[width=0.4\textwidth]{warping-1-6.png}
	}
	\caption{Вид функции смещения $d_2(x;a,b,c)$}
	\label{fig:warping-d2-example}
\end{figure}

\section{Методика поиска оптимальных функций смещения}

Проводился поиск оптимальных параметров рассматриваемых малопараметрических (одно"~ и двухпараметрических) моделей функций смещения для разных типов размытия, а затем "--- сравнительный анализ результатов обработки размытых изображений деформационным методом повышения резкости с использованием полученных результатов.

Для нахождения оптимальных параметров проводилась минимизация среднего значения показателя RMSE по всем изображениям для одного типа размытия:

\begin{align*}
	\frac{1}{\left|V\right|}\sum_{V}{RMSE\left(u,v\right) \rightarrow \min_{z}}, \quad
	RMSE\left(u,v\right)=\sqrt{\frac{1}{WH}\sum_{i,j}{(u_{ij}-v_{ij})}^2},
\end{align*}

%Для нахождения оптимальных параметров проводилась минимизация среднего значения показателя MSE по всем изображениям для одного типа размытия:
%
%\begin{align*}
%	\frac{1}{\left|V\right|}\sum_{V}{MSE\left(u,v\right) \rightarrow \min_{z}}, \quad
%	MSE\left(u,v\right)=\frac{1}{WH}\sum_{i,j}{(u_{ij}-v_{ij})}^2,
%\end{align*}

\noindent а в качестве показателя качества "--- значение $PSNR\left(u,v\right) = 10 \cdot \log_{10}{\left(\frac{255^2}{MSE\left(u,v\right)}\right)}$.

Здесь $v$ "--- размытое изображение, обработанное методом повышения резкости, $u$ "--- соответствующее ему истинное резкое изображение, $V$ "--- множество всех размытых изображений для одного ядра размытия, $W$ и $H$ "--- ширина и высота изображения в пикселях соответственно, $z$ "--- соответствующие рассматриваемой функции смещения параметры из множества $\left\{a, b, c, s\right\}$.

Для решения задачи минимизации применялся метод Нелдера"=Мида~\cite{10.1093/comjnl/7.4.308}. Число итераций было ограничено числом 100, минимальный шаг был установлен в $10^{-3}$. С начальным приближением $a=1$, $b=1.5$, $c=0.5$, $s=1$ процесс оптимизации завершился примерно на 50-й итерации. Было сделано несколько запусков процесса оптимизации с разными начальными приближениями для достижения глобального оптимума.

\section{Результаты экспериментов}

В ходе экспериментов было установлено, что наилучшие результаты достигаются при $a=-c$. Такое соотношение параметров обеспечивает наибольшее смещение пикселей в районе контура без нарушения ограничения на производную функции смещения. Поэтому в дальнейшем это соотношение всегда выполняется, что позволяет перейти к модели с двумя (функция $d_2(x;a,b,c=-a)$) и одним (функция $d_1(x;a,c=-a)$) параметрами.

%После проведения исследования и поиска оптимальных параметров функций смещения для каждого типа размытия были получены следующие результаты. Условные обозначения типов ядер размытия: Г "--- гауссово ядро, К "--- <<круг>>, КЦ "--- <<круг с кольцом>>.

%На Рис.~\ref{fig:warping-best-displacements} представлены функции смещения, оптимальные только для своего типа размытия. Красным цветом отмечена модель с двумя параметрами (с функцией $d_2\left(x\right)$), а синим "--- модель с одним параметром  (с функцией $d_1\left(x\right)$). Условные обозначения типов ядер размытия: Г "--- гауссово ядро, К "--- круг, КЦ "--- круг с кольцом.

%\begin{figure}[ht]
%	\centerfloat{
%		\includegraphics[width=0.9\textwidth]{warping-1-7.png}
%	}
%	\caption{Функция смещения для моделей с 1 и 2 параметрами}
%	\label{fig:warping-best-displacements}
%\end{figure}

%Усреднённые по набору изображений значения RMSE входных размытых и обработанных изображений для функций смещения $d_0\left(x\right)$ и $d_2\left(x\right)$ с оптимальными параметрами приведены в Табл.~\ref{tab:warping-rmse-d0-d2}. Из неё видно, что предложенная функция $d_2\left(x\right)$ позволяет достичь более качественного результата.

%\begin{table} [htbp]%
%%	\tiny
%	\small
%	\centering
%	\caption{Средние значения RMSE размытых и обработанных изображений для функций смещения $d_0\left(x\right)$ и $d_2\left(x\right)$ (сила размытия обозначена номером радиуса размытия в порядке возрастания)}%
%	\label{tab:warping-rmse-d0-d2}% label всегда желательно идти после caption
%	\renewcommand{\arraystretch}{1.5}%% Увеличение расстояния между рядами, для улучшения восприятия.
%	\begin{SingleSpace}
%		\begin{tabulary}{\textwidth}{@{}@{\extracolsep{5pt}}cCCCCCCCCC@{}} %Вертикальные полосы не используются принципиально, как и лишние горизонтальные (допускается по ГОСТ 2.105 пункт 4.4.5) % @{} позволяет прижиматься к краям
%			\toprule     %%% верхняя линейка
%			& \multicolumn{3}{@{}c@{}}{\makecell{Размытые \\ изображения}} & \multicolumn{3}{@{}c@{}}{\makecell{$d_0\left(x\right)$}} & \multicolumn{3}{@{}c@{}}{\makecell{$d_2\left(x\right)$}} \\
%			\cmidrule(r){2-4}\cmidrule(lr){5-7}\cmidrule(l){8-10}
%			\makecell{Сила \\ размытия}  &  Г & К & КЦ  &  Г & К & КЦ  &  Г & К & КЦ \\
%			\midrule %%% тонкий разделитель. Отделяет названия столбцов. Обязателен по ГОСТ 2.105 пункт 4.4.5
%			1 & 13.470 & 9.750 & 11.004 & 13.119 & 9.603 & 10.794 & 13.015 & 9.516 & 10.620 \\
%			2 & 15.481 & 11.849 & 13.525 & 15.039 & 11.585 & 13.176 & 14.912 & 11.492 & 12.974 \\
%			3 & 16.905 & 13.869 & 15.446 & 16.385 & 13.483 & 14.980 & 16.260 & 13.365 & 14.750 \\
%			4 & 18.037 & 14.975 & 16.383 & 17.489 & 14.558 & 15.898 & 17.370 & 14.455 & 15.710 \\
%			5 & 18.951 & 16.093 & 17.488 & 18.386 & 15.622 & 16.954 & 18.274 & 15.525 & 16.782 \\
%			6 & 19.749 & 16.893 & 18.255 & 19.169 & 16.412 & 17.712 & 19.063 & 16.327 & 17.563 \\
%			7 & 20.437 & 17.625 & 18.973 & 19.836 & 17.120 & 18.405 & 19.730 & 17.044 & 18.273 \\
%			8 & 21.070 & 18.270 & 19.599 & 20.462 & 17.761 & 19.027 & 20.352 & 17.693 & 18.905 \\
%			\midrule
%			Среднее & 18.013 & 14.916 & 16.334 & 17.486 & 14.518 & 15.868 & 17.372 & 14.427 & 15.697 \\
%			\bottomrule %%% нижняя линейка
%		\end{tabulary}%
%	\end{SingleSpace}
%\end{table}

Усреднённые по набору изображений значения PSNR входных размытых и обработанных изображений для функций смещения $d_0\left(x\right)$ и $d_2\left(x\right)$ с оптимальными параметрами приведены в Табл.~\ref{tab:warping-psnr-d0-d2}. Из неё видно, что предложенная функция $d_2\left(x\right)$ позволяет достичь более качественного результата. Условные обозначения типов ядер размытия: Г "--- гауссово ядро, К "--- <<круг>>, КЦ "--- <<круг с кольцом>>.

%\begin{table} [htbp]%
%	%	\tiny
%	\small
%	\centering
%	\caption{Средние значения PSNR размытых и обработанных изображений для функций смещения $d_0\left(x\right)$ и $d_2\left(x\right)$ (сила размытия обозначена номером значения параметра размытия в порядке возрастания)}%
%	\label{tab:warping-psnr-d0-d2}% label всегда желательно идти после caption
%	\renewcommand{\arraystretch}{1.5}%% Увеличение расстояния между рядами, для улучшения восприятия.
%	\begin{SingleSpace}
%		\begin{tabulary}{\textwidth}{@{}@{\extracolsep{5pt}}cCCCCCCCCC@{}} %Вертикальные полосы не используются принципиально, как и лишние горизонтальные (допускается по ГОСТ 2.105 пункт 4.4.5) % @{} позволяет прижиматься к краям
%			\toprule     %%% верхняя линейка
%			& \multicolumn{3}{@{}c@{}}{\makecell{Размытые \\ изображения}} & \multicolumn{3}{@{}c@{}}{\makecell{$d_0\left(x\right)$}} & \multicolumn{3}{@{}c@{}}{\makecell{$d_2\left(x\right)$}} \\
%			\cmidrule(r){2-4}\cmidrule(lr){5-7}\cmidrule(l){8-10}
%			\makecell{Сила \\ размытия}  &  Г & К & КЦ  &  Г & К & КЦ  &  Г & К & КЦ \\
%			\midrule %%% тонкий разделитель. Отделяет названия столбцов. Обязателен по ГОСТ 2.105 пункт 4.4.5
%			1	& 25.543	& 28.351	& 27.300	& 25.773	& 28.483	& 27.468	& 25.842	& 28.561	& 27.608 \\
%			2	& 24.335	& 26.657	& 25.508	& 24.587	& 26.853	& 25.735	& 24.660	& 26.923	& 25.870 \\
%			3	& 23.571	& 25.290	& 24.355	& 23.842	& 25.535	& 24.621	& 23.908	& 25.612	& 24.755 \\
%			4	& 23.008	& 24.623	& 23.843	& 23.276	& 24.869	& 24.104	& 23.335	& 24.930	& 24.207 \\
%			5	& 22.578	& 23.998	& 23.276	& 22.841	& 24.256	& 23.546	& 22.894	& 24.310	& 23.634 \\
%			6	& 22.220	& 23.577	& 22.903	& 22.478	& 23.827	& 23.165	& 22.527	& 23.873	& 23.239 \\
%			7	& 21.922	& 23.208	& 22.568	& 22.182	& 23.461	& 22.832	& 22.228	& 23.499	& 22.895 \\
%			8	& 21.658	& 22.896	& 22.286	& 21.912	& 23.141	& 22.543	& 21.959	& 23.175	& 22.599 \\
%			\midrule
%			Среднее	& 23.104	& 24.825	& 24.005	& 23.361	& 25.053	& 24.252	& 23.419	& 25.110	& 24.351  \\
%			\bottomrule %%% нижняя линейка
%		\end{tabulary}%
%	\end{SingleSpace}
%\end{table}

\begin{sidewaystable} [htbp]%
	\centering
	\caption{Средние значения PSNR размытых и обработанных изображений для функций смещения $d_0\left(x\right)$ и $d_2\left(x\right)$ (сила размытия обозначена номером значения параметра размытия в порядке возрастания)}%
	\label{tab:warping-psnr-d0-d2}% label всегда желательно идти после caption
	\renewcommand{\arraystretch}{1.5}%% Увеличение расстояния между рядами, для улучшения восприятия.
	\begin{SingleSpace}
		\begin{tabulary}{\textwidth}{@{}@{\extracolsep{10pt}}cCCCCCCCCC@{}} %Вертикальные полосы не используются принципиально, как и лишние горизонтальные (допускается по ГОСТ 2.105 пункт 4.4.5) % @{} позволяет прижиматься к краям
			\toprule     %%% верхняя линейка
			& \multicolumn{3}{@{}c@{}}{\makecell{Размытые \\ изображения}} & \multicolumn{3}{@{}c@{}}{\makecell{$d_0\left(x\right)$}} & \multicolumn{3}{@{}c@{}}{\makecell{$d_2\left(x\right)$}} \\
			\cmidrule(r){2-4}\cmidrule(lr){5-7}\cmidrule(l){8-10}
			\makecell{Сила \\ размытия}  &  Г & К & КЦ  &  Г & К & КЦ  &  Г & К & КЦ \\
			\midrule %%% тонкий разделитель. Отделяет названия столбцов. Обязателен по ГОСТ 2.105 пункт 4.4.5
			1	& 25.543	& 28.351	& 27.300	& 25.773	& 28.483	& 27.468	& 25.842	& 28.561	& 27.608 \\
			2	& 24.335	& 26.657	& 25.508	& 24.587	& 26.853	& 25.735	& 24.660	& 26.923	& 25.870 \\
			3	& 23.571	& 25.290	& 24.355	& 23.842	& 25.535	& 24.621	& 23.908	& 25.612	& 24.755 \\
			4	& 23.008	& 24.623	& 23.843	& 23.276	& 24.869	& 24.104	& 23.335	& 24.930	& 24.207 \\
			5	& 22.578	& 23.998	& 23.276	& 22.841	& 24.256	& 23.546	& 22.894	& 24.310	& 23.634 \\
			6	& 22.220	& 23.577	& 22.903	& 22.478	& 23.827	& 23.165	& 22.527	& 23.873	& 23.239 \\
			7	& 21.922	& 23.208	& 22.568	& 22.182	& 23.461	& 22.832	& 22.228	& 23.499	& 22.895 \\
			8	& 21.658	& 22.896	& 22.286	& 21.912	& 23.141	& 22.543	& 21.959	& 23.175	& 22.599 \\
			\midrule
			Среднее	& 23.104	& 24.825	& 24.005	& 23.361	& 25.053	& 24.252	& 23.419	& 25.110	& 24.351  \\
			\bottomrule %%% нижняя линейка
		\end{tabulary}%
	\end{SingleSpace}
\end{sidewaystable}

%В Табл.~\ref{tab:warping-rmse-d1} представлены показатели снижения RMSE относительно размытых изображений для вариантов алгоритма на основе функции смещения с 2 и с 1 параметром. Модуль относительной разницы не превышает 4\%, а в среднем равен 1.2\%, что позволяет говорить о сравнимом качестве работы обоих вариантов метода и возможности использовать однопараметрический вариант без значительных потерь в качестве.
%
%\begin{table} [htbp]%
%	\centering
%	\caption{Средние значения снижения RMSE для изображений, обработанных с использованием функции смещения $d_2\left(x\right)$ и $d_1\left(x\right)$, по сравнению с размытыми изображениями}%
%	\label{tab:warping-rmse-d1}% label всегда желательно идти после caption
%	\renewcommand{\arraystretch}{1.5}%% Увеличение расстояния между рядами, для улучшения восприятия.
%	\begin{SingleSpace}
%		\begin{tabulary}{\textwidth}{@{}@{\extracolsep{10pt}}cCCCCCC@{}} %Вертикальные полосы не используются принципиально, как и лишние горизонтальные (допускается по ГОСТ 2.105 пункт 4.4.5) % @{} позволяет прижиматься к краям
%			\toprule     %%% верхняя линейка
%			& \multicolumn{3}{@{}c@{}}{\makecell{$d_2\left(x\right)$}} & \multicolumn{3}{@{}c@{}}{\makecell{$d_1\left(x\right)$}} \\
%			\cmidrule(r){2-4}\cmidrule(l){5-7}
%			\makecell{Сила \\ размытия}  &  Г & К & КЦ  &  Г & К & КЦ \\
%			\midrule %%% тонкий разделитель. Отделяет названия столбцов. Обязателен по ГОСТ 2.105 пункт 4.4.5
%			1 & 0.455 & 0.234 & 0.384 & 0.456 & 0.227 & 0.399 \\
%			2 & 0.569 & 0.357 & 0.551 & 0.569 & 0.365 & 0.551 \\
%			3 & 0.645 & 0.504 & 0.696 & 0.645 & 0.508 & 0.671 \\
%			4 & 0.667 & 0.520 & 0.673 & 0.666 & 0.518 & 0.654 \\
%			5 & 0.677 & 0.568 & 0.706 & 0.677 & 0.563 & 0.693 \\
%			6 & 0.686 & 0.566 & 0.692 & 0.686 & 0.56 & 0.682 \\
%			7 & 0.707 & 0.581 & 0.700 & 0.708 & 0.578 & 0.687 \\
%			8 & 0.718 & 0.577 & 0.694 & 0.718 & 0.575 & 0.676 \\
%			\midrule
%			Среднее & 0.641 & 0.489 & 0.637 & 0.641 & 0.487 & 0.627 \\
%			\bottomrule %%% нижняя линейка
%		\end{tabulary}%
%	\end{SingleSpace}
%\end{table}

В Табл.~\ref{tab:warping-psnr-d1} представлены показатели среднего прироста PSNR (ISNR) относительно размытых изображений для вариантов алгоритма на основе функции смещения с 2 и с 1 параметром. Модуль относительной разницы не превышает 4.2\%, а в среднем равен 1.1\%, что позволяет говорить о сравнимом качестве работы обоих вариантов метода и возможности использовать однопараметрический вариант без значительных потерь в качестве.
Оптимальные значения параметра $a$ однопараметрического метода с функцией $d_1\left(x\right)$ для гауссова ядра размытия, ядра типа <<круг>> и ядра типа <<круг с кольцом>> составили $1.28$, $1.12$ и $1.16$ соответственно.
%Оптимальные значения параметра $a$ однопараметрического метода с функцией $d_1\left(x\right)$ для гауссова ядра размытия, ядра типа <<круг>> и ядра типа <<круг с кольцом>> составили $1.92$, $1.12$ и $1.16$ соответственно.

\begin{table} [htbp]%
	\centering
	\caption{Средние значения прироста PSNR для изображений, обработанных с использованием функции смещения $d_2\left(x\right)$ и $d_1\left(x\right)$, по сравнению с размытыми изображениями}%
	\label{tab:warping-psnr-d1}% label всегда желательно идти после caption
	\renewcommand{\arraystretch}{1.5}%% Увеличение расстояния между рядами, для улучшения восприятия.
	\begin{SingleSpace}
		\begin{tabulary}{\textwidth}{@{}@{\extracolsep{10pt}}cCCCCCC@{}} %Вертикальные полосы не используются принципиально, как и лишние горизонтальные (допускается по ГОСТ 2.105 пункт 4.4.5) % @{} позволяет прижиматься к краям
			\toprule     %%% верхняя линейка
			& \multicolumn{3}{@{}c@{}}{\makecell{$d_2\left(x\right)$}} & \multicolumn{3}{@{}c@{}}{\makecell{$d_1\left(x\right)$}} \\
			\cmidrule(r){2-4}\cmidrule(l){5-7}
			\makecell{Сила \\ размытия}  &  Г & К & КЦ  &  Г & К & КЦ \\
			\midrule %%% тонкий разделитель. Отделяет названия столбцов. Обязателен по ГОСТ 2.105 пункт 4.4.5
			1	& 0.299	& 0.210	& 0.308	& 0.299	& 0.204	& 0.321 \\
			2	& 0.325	& 0.266	& 0.362	& 0.325	& 0.272	& 0.361 \\
			3	& 0.337	& 0.322	& 0.400	& 0.337	& 0.324	& 0.386 \\
			4	& 0.327	& 0.307	& 0.364	& 0.327	& 0.306	& 0.354 \\
			5	& 0.316	& 0.312	& 0.358	& 0.316	& 0.309	& 0.351 \\
			6	& 0.307	& 0.296	& 0.336	& 0.307	& 0.293	& 0.331 \\
			7	& 0.306	& 0.291	& 0.327	& 0.306	& 0.290	& 0.321 \\
			8	& 0.301	& 0.279	& 0.313	& 0.301	& 0.278	& 0.305 \\
			\midrule
			Среднее	& 0.315	& 0.285	& 0.346	& 0.315	& 0.285	& 0.341 \\
			\bottomrule %%% нижняя линейка
		\end{tabulary}%
	\end{SingleSpace}
\end{table}

%Из Рис.~\ref{fig:warping-best-displacements} видно, что для разных типов ядер размытия относительные значения оптимальных параметров функций смещения близки.
Дополнительный анализ изменения среднего ISNR (прироста показателя PSNR) показал, что при несильном отклонении параметра от оптимального значения качество результирующего изображения снижается слабо (см. Рис.~\ref{fig:warping-isnr-change}).

\begin{figure}[ht]
	\centerfloat{
		\includegraphics[width=0.8\textwidth]{warping-1-extra1.png}
	}
	\caption{Изменение ISNR (прироста показателя PSNR) при относительном изменении параметра $a$}
	\label{fig:warping-isnr-change}
\end{figure}

На Рис.~\ref{fig:warping-eye}~и~\ref{fig:warping-eye2} приведены примеры результата обработки медицинского изображения полученным методом. Видно повышение резкости контуров сосудов и диска зрительного нерва.

\begin{figure}[ht]
	\centerfloat{
		\hfill
		\subcaptionbox[List-of-Figures entry]{Входное изображение}{%
			\includegraphics[height=0.3\textheight]{warping-1-8a.png}}
		\hfill
		\subcaptionbox{Обработанное изображение}{%
			\includegraphics[height=0.3\textheight]{warping-1-8b.png}}
		\hfill
	}
	\caption{Применение деформационного алгоритма с использованием функции смещения $d_1\left(x\right)$ как шага постобработки в задаче повышения разрешения медицинских изображений}
	\label{fig:warping-eye}
\end{figure}

\begin{figure}[ht]
	\centerfloat{
		\hfill
		\subcaptionbox[List-of-Figures entry]{Входное изображение}{%
			\includegraphics[height=0.25\textheight]{warping-1-extra2a.png}}
		\hfill
		\subcaptionbox{Обработанное изображение}{%
			\includegraphics[height=0.25\textheight]{warping-1-extra2b.png}}
		\hfill
	}
	\caption{Применение деформационного алгоритма с использованием функции смещения $d_1\left(x\right)$ как шага постобработки в задаче повышения разрешения медицинских изображений}
	\label{fig:warping-eye2}
\end{figure}

\section{Выводы} 

%Предложенные функции смещения $d_2\left(x\right)$ и $d_1\left(x\right)$ демонстрируют улучшение качества обработки изображений по сравнению с оригинальной функцией $d_0\left(x\right)$, выражаемое в приросте снижения показателя RMSE, в среднем на 30\%. Вариант алгоритма на основе однопараметрической функция смещения $d_1\left(x\right)$ почти не отличается от двухпараметрического варианта качеством результатов, благодаря этому можно использовать его для всех трёх рассмотренных ядер размытия.

В рамках исследования в данный главе разработан малопараметрический метод повышения резкости медицинских изображений на основе деформации пиксельной сетки для различных математических моделей оптического размытия изображений. 
Предложенные для метода функции смещения $d_2\left(x\right)$ и $d_1\left(x\right)$ демонстрируют улучшение качества обработки изображений по сравнению с оригинальной функцией $d_0\left(x\right)$, выражаемое в приросте показателя PNSR, в среднем на 27\%. Вариант алгоритма на основе однопараметрической функция смещения $d_1\left(x\right)$ почти не отличается от двухпараметрического варианта качеством результатов, благодаря чему его можно использовать для всех трёх рассмотренных ядер размытия.

\FloatBarrier
