
{\actuality} 

В настоящее время в разных областях биологии и медицины повсеместно используются цифровые изображения и развиваются алгоритмы их обработки, в том числе и на основе глубокого обучения. Несмотря на быстрый научно"=технический прогресс, в силу различных причин оборудование, используемое для получения таких изображений, не всегда способно решить возникающие задачи, однако применение математических методов позволяет приблизиться к их решению и повысить качество уже полученных данных. Поэтому разработка алгоритмов повышения качества изображений с учётом современных условий остаётся актуальной.

Изображения клеточных структур, возникающие в области флуоресцентной микроскопии, являются одним из классов изображений, увеличение разрешения которых необходимо на практике. Одними из примеров таких задач являются ситуации, при которых разрешающая способность оптической системы микроскопа упирается в теоретический предел из-за дифракции света или когда увеличение оптической системы микроскопа не позволяет реализовать весь потенциал теоретической разрешающей способности.

Хотя появились методики преодоления дифракционного предела с использованием специального оборудования, на практике более доступным подходом является использование специальных стохастический мигающих красителей. Серия снимков, на которых молекулы красителя имеют разную яркость, преобразуется в изображение повышенного разрешения. Благодаря повышенному по сравнению с одним снимком количеству информации и слабой корреляции мигания молекул флуорофора, вычислительные методы позволяют существенно повысить разрешающие способности микроскопов без дорогостоящей модификации оборудования. В то же время из-за необходимости получать большое число изображений за короткий срок, повышается значимость таких параметров, как уровень шума и скорость выгорания красителей.

Для макроскопических же изображений в медицине по причине строгих ограничений со стороны закона и специфики области, связанной с повышенными рисками ошибок, важен вопрос аккуратной обработки изображений для избежания потери или привнесения информации в имеющиеся данные и контроля качества получаемых изображений. Очень актуальна задача повышения резкости изображений. А для методов глубокого обучения в медицинской диагностике, которые внедряются в практику во всём мире, является анализ и предобработка входных данных. Необходим контроль соответствия входной информации и применяемого обученного или обучаемого алгоритма глубокого обучения.

При изучении рентгеновских снимков и, в частности, при диагностике туберкулёза лёгких важным фактором является жёсткость снимка, так как она напрямую влияет на его информативность: на изображении должны в достаточном количестве присутствовать важные для вынесения верного решения детали. Кроме того, необходима правильная настройка контрастности снимков, чтобы различия изображений из обучающего набора и тестовых снимков, которые получаются на различном оборудовании в разных условиях, не были велики настолько, чтобы влиять на качество работы алгоритмов диагностики.

Основное внимание в данной работе уделено повышению качества изображений клеточных структур и анализу качества медицинских изображений. Под качеством понимается как наличие у изображения определённых характеристик (резкость, контрастность, разрешение), так и их степень сходства с эталонными изображениями, для работы с которыми оптимизирован алгоритм машинного или глубокого обучения.

% {\progress}
% Этот раздел должен быть отдельным структурным элементом по
% ГОСТ, но он, как правило, включается в описание актуальности
% темы. Нужен он отдельным структурынм элемементом или нет ---
% смотрите другие диссертации вашего совета, скорее всего не нужен.

{\aim}

Цель данной работы состоит в разработке адаптивных методов анализа и обработки биомедицинских изображений различных модальностей на основе методов математического моделирования, их алгоритмическая и программная реализация для решения задач повышения разрешения изображений флуоресцентной микроскопии, повышения резкости и контроля качества медицинских изображений.


{\novelty}

В данной работе были получены новые методы и алгоритмы:
\begin{enumerate}[beginpenalty=10000]
 	\item регуляризирующий метод повышения разрешения и резкости изображений флуоресцентной мигающей микроскопии;

	\item малопараметрический деформационный метод повышения резкости изображений;

	\item нейросетевой метод контроля качества рентгеновских снимков грудной клетки, основанный на автоматическом анализе жёсткости рентгенограммы;
	
	\item алгоритм компьютерной диагностики туберкулёза лёгких по рентгеновскому снимку грудной клетки.
\end{enumerate}

{\influence}

Создан программный комплекс повышения разрешения изображений флуоресцентной мигающей микроскопии, повышения резкости медицинских изображений и определения качества рентгенограмм грудной клетки для задачи диагностики туберкулёза лёгких.

Предложенные методы могут применяться как отдельно, так и в составе других систем анализа и повышения качества биомедицинских изображений, в качестве вспомогательных инструментов в работе учёных-биологов и врачей, а также могут быть модифицированы для применения в иных областях медицины и биологии, отличных от рассмотренных в работе.

{\progress}

Исследование, проведённое в данной работе, затрагивает три различных области анализа и повышения качества биомедицинских изображений.

В области флуоресцентной микроскопии, получившей распространение благодаря таким своим достоинствам, как высокий контраст получаемых изображений, возможность избирательного окрашивания структур разной природы в разные цвета и наблюдения за живыми образцами, к настоящему времени был разработан ряд методов повышения разрешающей способности микроскопов. Некоторые из них (конфокальная микроскопия, STED, SIM) требуют особого оборудования, другие же (STORM, PALM, SOFI, SRRF, MUSICAL) появились уже в 21 веке и полагаются на использование специальных мигающих красителей и вычислительные алгоритмы.

В прошлом и в текущем веке было разработано множество эффективных методов повышения резкости и восстановления смазанных изображений. Многие из них требуют указания типа размытия для работы, и хотя за последние 20 лет появились алгоритмы автоматического определения размытия по входному изображению, чувствительность методов повышения резкости к шуму и ошибкам в определении и указании типа и силы размытия остаётся значимым фактором. Метод повышения резкости изображений с помощью деформации пиксельной сетки не предназначен для решения задачи полноценного восстановления смазанных изображений, но может служить в качестве этапа постобработки восстановленного другим алгоритмом изображения, повышая резкость без порождения артефактов на изображении. Выбор оптимальной функции смещения пикселей позволяет регулировать эффективность обработки изображения.

С новым витком развития области науки в середине 2010"~х годов и повсеместном распространении алгоритмов на основе искусственного интеллекта и глубокого обучения вопрос о влиянии качества входных данных на результаты обработки и о контроле качества этих данных снова получил высокую актуальность. За последнее десятилетие были проведены исследования на тему целенаправленных атак на алгоритмы анализа данных с целью повлиять на результаты их работы путём незаметного для человека изменения входных данных и защиты от таких атак. В то же время в области медицинской диагностики заболеваний по рентгенограммам грудной клетки за последние 10 лет были разработаны методы контроля некоторых условий съёмки, изучено влияние качества снимков на точность диагностики некоторых заболеваний.


{\methods} 

В основе методологии исследования лежат методы математического моделирования в обработке и анализе изображений, ряд вычислительных экспериментов реализован в рамках задач машинного обучения и анализа изображений с помощью искусственных и реальных данных.


{\reliability}

Достоверность результатов проведённых исследований обеспечивается опорой на теоретическую базу, математической обоснованностью разработанных методов, воспроизводимыми вычислительными экспериментами и тестированием алгоритмов на искусственных и реальных данных. Значительная часть данных, использованных для создания и тестирования разработанных методов, находится в открытом доступе.


{\probation}

Основные результаты работы докладывались на:

\begin {enumerate}[beginpenalty=10000]
	\item 7-й Международной конференции по теории обработки изображений, методам и применениям <<IPTA 2017>> (Монреаль, Канада, 2017);
	
	\item 7-м Европейском семинаре по обработке визуальной информации <<EUVIP~2018>> (Тампере, Финляндия, 2018);
	
	\item 4"~й Международной конференции по обработке биомедицинских изображений и сигналов <<ICBSP~2019>> (Нагоя, Япония, 2019);
	
	\item Всероссийской конференции <<Ломоносовские чтения"~2020>> (Москва, 2020);
	
	\item 6"~й Международной конференции по обработке биомедицинских изображений и сигналов <<ICBSP~2021>> (Сямынь, Китай, 2021);
\end {enumerate}


\ifnumequal{\value{bibliosel}}{0}
{%%% Встроенная реализация с загрузкой файла через движок bibtex8. (При желании, внутри можно использовать обычные ссылки, наподобие `\cite{vakbib1,vakbib2}`).
    {\publications} 
    
    По теме исследования опубликовано~8 работ, из них~6 работы в изданиях, рекомендуемых для защиты в диссертационном совете МГУ имени М.В.~Ломоносова по специальности 1.2.2.~<<Математическое моделирование, численные методы и комплексы программ>>,
    и~2 работы, опубликованные в сборниках трудов конференций.
     
}%
{%%% Реализация пакетом biblatex через движок biber
    \begin{refsection}[bl-author, bl-registered]
        % Это refsection=1.
        % Процитированные здесь работы:
        %  * подсчитываются, для автоматического составления фразы "Основные результаты ..."
        %  * попадают в авторскую библиографию, при usefootcite==0 и стиле `\insertbiblioauthor` или `\insertbiblioauthorgrouped`
        %  * нумеруются там в зависимости от порядка команд `\printbibliography` в этом разделе.
        %  * при использовании `\insertbiblioauthorgrouped`, порядок команд `\printbibliography` в нём должен быть тем же (см. biblio/biblatex.tex)
        %
        % Невидимый библиографический список для подсчёта количества публикаций:
        \printbibliography[heading=nobibheading, section=1, env=countauthorvak,          keyword=biblioauthorvak]%
        \printbibliography[heading=nobibheading, section=1, env=countauthorwos,          keyword=biblioauthorwos]%
        \printbibliography[heading=nobibheading, section=1, env=countauthorscopus,       keyword=biblioauthorscopus]%
        \printbibliography[heading=nobibheading, section=1, env=countauthorconf,         keyword=biblioauthorconf]%
        \printbibliography[heading=nobibheading, section=1, env=countauthorother,        keyword=biblioauthorother]%
        \printbibliography[heading=nobibheading, section=1, env=countregistered,         keyword=biblioregistered]%
        \printbibliography[heading=nobibheading, section=1, env=countauthorpatent,       keyword=biblioauthorpatent]%
        \printbibliography[heading=nobibheading, section=1, env=countauthorprogram,      keyword=biblioauthorprogram]%
        \printbibliography[heading=nobibheading, section=1, env=countauthor,             keyword=biblioauthor]%
        \printbibliography[heading=nobibheading, section=1, env=countauthorvakscopuswos, filter=vakscopuswos]%
        \printbibliography[heading=nobibheading, section=1, env=countauthorscopuswos,    filter=scopuswos]%
        %
        \nocite{*}%
        %
        {\publications}
        
    По теме исследования опубликовано~8 работ, из них~6 работ в изданиях, индексируемых системами Web~of~Science и Scopus, рекомендованных для защиты в диссертационном совете МГУ имени М.В.~Ломоносова по специальности 1.2.2.~<<Математическое моделирование, численные методы и комплексы программ>>, и~2 работы, опубликованные в иных изданиях.
    	
    	%, 3 работы в журналах, входящих в перечень изданий ВАК при Минобрнауки России
        %откомментить при использовании автоподсчета
%        Основные результаты по теме диссертации изложены в~\arabic{citeauthor}~печатных изданиях,
%        \arabic{citeauthorvak} из которых изданы в журналах, рекомендованных ВАК\sloppy%
%        \ifnum \value{citeauthorscopuswos}>0%
%            , \arabic{citeauthorscopuswos} "--- в~периодических научных журналах, индексируемых Web of~Science и Scopus\sloppy%
%        \fi%
%        \ifnum \value{citeauthorconf}>0%
%            , \arabic{citeauthorconf} "--- в~тезисах докладов.
%        \else%
%            .
%        \fi%
%        \ifnum \value{citeregistered}=1%
%            \ifnum \value{citeauthorpatent}=1%
%                Зарегистрирован \arabic{citeauthorpatent} патент.
%            \fi%
%            \ifnum \value{citeauthorprogram}=1%
%                Зарегистрирована \arabic{citeauthorprogram} программа для ЭВМ.
%            \fi%
%        \fi%
%        \ifnum \value{citeregistered}>1%
%            Зарегистрированы\ %
%            \ifnum \value{citeauthorpatent}>0%
%            \formbytotal{citeauthorpatent}{патент}{}{а}{}\sloppy%
%            \ifnum \value{citeauthorprogram}=0 . \else \ и~\fi%
%            \fi%
%            \ifnum \value{citeauthorprogram}>0%
%            \formbytotal{citeauthorprogram}{программ}{а}{ы}{} для ЭВМ.
%            \fi%
%        \fi%
        
        
        % К публикациям, в которых излагаются основные научные результаты диссертации на соискание учёной
        % степени, в рецензируемых изданиях приравниваются патенты на изобретения, патенты (свидетельства) на
        % полезную модель, патенты на промышленный образец, патенты на селекционные достижения, свидетельства
        % на программу для электронных вычислительных машин, базу данных, топологию интегральных микросхем,
        % зарегистрированные в установленном порядке.(в ред. Постановления Правительства РФ от 21.04.2016 N 335)
    \end{refsection}%
    \begin{refsection}[bl-author, bl-registered]
        % Это refsection=2.
        % Процитированные здесь работы:
        %  * попадают в авторскую библиографию, при usefootcite==0 и стиле `\insertbiblioauthorimportant`.
        %  * ни на что не влияют в противном случае

    \end{refsection}%
        %
        % Всё, что вне этих двух refsection, это refsection=0,
        %  * для диссертации - это нормальные ссылки, попадающие в обычную библиографию
        %  * для автореферата:
        %     * при usefootcite==0, ссылка корректно сработает только для источника из `external.bib`. Для своих работ --- напечатает "[0]" (и даже Warning не вылезет).
        %     * при usefootcite==1, ссылка сработает нормально. В авторской библиографии будут только процитированные в refsection=0 работы.
}


{\contribution} 

Все результаты работы получены автором лично под научным руководством д.ф.-м.н., проф. А.С. Крылова. В работах, написанных в соавторстве, вклад автора работы в полученные результаты математического моделирования, численные методы и разработку комплекса программ является определяющим.

{\defpositions}

\begin {enumerate}[beginpenalty=10000]
	\item Метод повышения разрешения изображений мигающей флуоресцентной микроскопии, основанный на итерационном процессе максимизации регуляризирующего функционала.
	
	\item Малопараметрический метод повышения резкости медицинских изображений на основе деформации пиксельной сетки.
	
	\item Метод автоматического контроля качества рентгенограмм грудной клетки для задач машинного обучения, основанный на анализе и классификации уровня жёсткости рентгеновского снимка грудной клетки.
	
	\item Программный комплекс повышения разрешения изображений мигающей флуоресцентной микроскопии, повышения резкости медицинских изображений, анализа качества рентгенограмм грудной клетки.
\end {enumerate}


