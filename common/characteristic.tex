
{\actuality} 

В настоящее время в разных областях биологии и медицины повсеместно используются цифровые изображения и развиваются алгоритмы их обработки, в том числе и на основе глубокого обучения. Несмотря на быстрый научно"=технический прогресс, в силу различных причин оборудование, используемое для получения таких изображений, не всегда способно решить возникающие задачи, однако применение математических методов позволяет приблизиться к их решению и повысить качество уже полученных данных. Поэтому разработка алгоритмов повышения качества изображений с учётом современных условий остаётся актуальной.

Изображения клеточных структур, возникающие в области флуоресцентной микроскопии, являются одним из классов изображений, увеличение разрешения которых необходимо на практике. Одними из примеров таких задач являются ситуации, при которых разрешающая способность оптической системы микроскопа упирается в теоретический предел из-за дифракции света или когда увеличение оптической системы микроскопа не позволяет реализовать весь потенциал теоретической разрешающей способности.

Хотя появились методики преодоления дифракционного предела с использованием специального оборудования, на практике более доступным подходом является использование специальных стохастический мигающих красителей. Серия снимков, на которых молекулы красителя имеют разную яркость, преобразуется в изображение повышенного разрешения. Благодаря повышенному по сравнению с одним снимком количеству информации и слабой корреляции мигания молекул флуорофора, вычислительные методы позволяют существенно повысить разрешающие способности микроскопов без дорогостоящей модификации оборудования. В то же время из-за необходимости получать большое число изображений за короткий срок, повышается значимость таких параметров, как уровень шума и скорость выгорания красителей.

Для макроскопических же изображений в медицине по причине строгих ограничений со стороны закона и специфики области, связанной с повышенными рисками ошибок, важен вопрос аккуратной обработки изображений для избежания потери или привнесения информации в имеющиеся данные и контроля качества получаемых изображений. По-прежнему остаётся актуальной более стандартная задача повышения резкости. А для методов глубокого обучения в медицинской диагностике, которые внедряются в практику во всём мире, является анализ и предобработка входных данных. Необходим контроль соответствия входной информации и применяемого обученного или обучаемого алгоритма глубокого обучения.

При изучении рентгеновских снимков и, в частности, при диагностике туберкулёза важным фактором является жёсткость снимка, так как она напрямую влияет на его информативность: на изображении должны в достаточном количестве присутствовать важные для вынесения верного решения детали. Кроме того, необходима правильная настройка контрастности снимков, чтобы различия изображений из обучающего набора и тестовых снимков, которые получаются на различном оборудовании в разных условиях, не были велики настолько, что влияют на качество работы алгоритмов диагностики.

Основное внимание в данной научно"=квалификационной работе уделено повышению качества изображений клеточных структур и анализу качества медицинских изображений. Под качеством понимается как наличие у изображения определённых характеристик (резкость, контрастность, разрешение), так и их степень сходства с эталонными изображениями, для работы с которыми оптимизирован алгоритм машинного или глубокого обучения.

% {\progress}
% Этот раздел должен быть отдельным структурным элементом по
% ГОСТ, но он, как правило, включается в описание актуальности
% темы. Нужен он отдельным структурынм элемементом или нет ---
% смотрите другие диссертации вашего совета, скорее всего не нужен.

{\aim}

Цель данной работы состоит в разработке адаптивных методов анализа и обработки биомедицинских изображений различных модальностей на основе методов математического моделирования, их алгоритмическая и программная реализация для решения задач повышения разрешения изображений флуоресцентной микроскопии, повышения резкости и контроля качества медицинских изображений.


{\novelty}

В данной работе были разработаны:

\begin{enumerate}[beginpenalty=10000]
 	\item регуляризирующий метод повышения разрешения и резкости изображений флуоресцентной мигающей микроскопии;

	\item однопараметрический метод деформационного метода повышения резкости изображений;

	\item нейросетевой метод контроля качества рентгеновских снимков грудной клетки, основанный на визуальном определении жёсткости рентгенограммы;

	\item алгоритм классификации рентгенограмм грудной клетки для диагностики туберкулёза;
	
	\item алгоритм компьютерной диагностики туберкулёза по рентгеновскому снимку грудной клетки.
\end{enumerate}

{\influence}

Создан программный комплекс повышения разрешения изображений флуоресцентной мигающей микроскопии, повышения резкости медицинских изображений и определения качества рентгенограмм грудной клетки для задачи диагностики туберкулёза. Комплекс способен обрабатывать изображения раздела флуоресцентной микроскопии и различных медицинских областей.

Предложенные методы могут применяться как отдельно, так и в составе других систем анализа и повышения качества биомедицинских изображений, в качестве вспомогательных инструментов в работе учёных-биологов и врачей.

{\progress}
%TODO
TODO: Написать.
%Исследование, проведенное в данной диссертационной работе, затрагивает три различных области обработки и анализа изображений. При обработке изображений иммунофлюоресцентной микроскопии тканей кожи и анализе качетства снимков рентгенографии используется модель хребтовых структур. Понятие хребтовых структур для цифровых изображений было введено  Р.М.~Хараликом в 1983 году. Применение дескрипторов, основанных на хребтовых структурах, к анализу медицинских изображений было изучено в работах С.М.~Пайзера и его коллег. Позже в 1998 году Т. Линдеберг разработал метод детектирования $\gamma$-нормализованных хребтовых структур на основе локальной максимизации соответствующим образом нормализованных собственных значений матрицы Гессе. Данная методика позже была применены К.~Стегером для обнаружения дорог на карте, К.~Франги для сегментации кровеносных сосудов, а также для обнаружению криволинейных и трубчатых структур на снимках (Ю.~Сато).
%
%Специфика задач, рассматриваемых в данной диссертации, состоит в том, что рассматривается применение хребтовых структур для обнаружения межклеточных границ и костных структур. Обнаруженные при помощи анализа хребтовых структур особенности применяются в связке с алгоритмами машинного обучения.
%
%В данной диссертационной работе также рассмотрены задачи подавления шума на изображениях. Это одна из самых старых и наиболее исследованных областей обработки цифровых сигналов. К сожалению, проблема объективной оценки качества шумоподавления до сих пор полностью не решена. Здесь стоит обратить внимание на работы Ч.~Ванг и А.~Бовика, которые занимались разработкой индекса структурного сходства для сравнения и оценки качества изображений. В даной работе было предложено развитие идей индекса структурного сходства и разработан метод шумоподавления, основанный на модифицированной версии этого индекса.  


{\methods} 

В основе методологии исследования лежат методы математического моделирования в обработке и анализе изображений, ряд вычислительных экспериментов реализован в рамках задач машинного обучения и анализа изображений с помощью искусственных и реальных данных.

{\reliability}
%TODO
TODO: Написать.
%Достоверность результатов обеспечивается воспроизводимыми численными экспериментами на искусственных и реальных данных. При этом, большая часть наборов данных, на которых производилась обучение и тестирование моделей находятся в открытом доступе. Так же производилось сравнение результатов, полученных с помощью разработанных методов, и разметки, сделанной медицинскими специалистами. 

{\probation}
%TODO begin
TODO: Добавить из презентации к предзащите.
%Основные результаты работы докладывались на:
%
%\begin {enumerate}[beginpenalty=10000]
%	\item ХХХIII научно-практической конференции c международным участием, посвященной 170-летию со дня рождения А.И. Поспелова "<Рахмановские чтения: от дерматологии А.И. Поспелова до наших дней --- 170 лет"> (Москва, Россия, 2016);	
%
%	\item 26-ой международной конференции по компьютерной графике и зрению "<ГрафиКон'2016"> (Нижний Новгород, Россия, 2016);
%
%	\item ХХХIV научно-практической конференции c международным участием "<Рахмановские чтения"> (Москва, Россия, 2017);
%
%	\item 9-ой международной конференции по компьютерной графике и обработке изображений "<ICGIP"> (Циндао, Китай, 2017);
%
%	\item 28-ой международной конференции по компьютерной графике и зрению "<ГрафиКон'2018"> (Томск, Россия, 2018);
%
%	\item Научной конференции "<Тихоновские чтения 2021"> (Москва, Россия, 2021);
%
%	\item 6-ой международной конференции по биомедицинской визуализации и обработке сигналов "<ICBSP 2021"> (Сямэнь, Китай, 2021);				
%\end {enumerate}
%TODO end

\ifnumequal{\value{bibliosel}}{0}
{%%% Встроенная реализация с загрузкой файла через движок bibtex8. (При желании, внутри можно использовать обычные ссылки, наподобие `\cite{vakbib1,vakbib2}`).
    {\publications} 
    
    По теме исследования опубликовано~X работ, из них~X работы в изданиях, рекомендуемых для защиты в диссертационном совете МГУ имени М.В.~Ломоносова по специальности 05.13.18,
    и~X работа, опубликованная в сборниках трудов конференций.
     
}%
{%%% Реализация пакетом biblatex через движок biber
    \begin{refsection}[bl-author, bl-registered]
        % Это refsection=1.
        % Процитированные здесь работы:
        %  * подсчитываются, для автоматического составления фразы "Основные результаты ..."
        %  * попадают в авторскую библиографию, при usefootcite==0 и стиле `\insertbiblioauthor` или `\insertbiblioauthorgrouped`
        %  * нумеруются там в зависимости от порядка команд `\printbibliography` в этом разделе.
        %  * при использовании `\insertbiblioauthorgrouped`, порядок команд `\printbibliography` в нём должен быть тем же (см. biblio/biblatex.tex)
        %
        % Невидимый библиографический список для подсчёта количества публикаций:
        \printbibliography[heading=nobibheading, section=1, env=countauthorvak,          keyword=biblioauthorvak]%
        \printbibliography[heading=nobibheading, section=1, env=countauthorwos,          keyword=biblioauthorwos]%
        \printbibliography[heading=nobibheading, section=1, env=countauthorscopus,       keyword=biblioauthorscopus]%
        \printbibliography[heading=nobibheading, section=1, env=countauthorconf,         keyword=biblioauthorconf]%
        \printbibliography[heading=nobibheading, section=1, env=countauthorother,        keyword=biblioauthorother]%
        \printbibliography[heading=nobibheading, section=1, env=countregistered,         keyword=biblioregistered]%
        \printbibliography[heading=nobibheading, section=1, env=countauthorpatent,       keyword=biblioauthorpatent]%
        \printbibliography[heading=nobibheading, section=1, env=countauthorprogram,      keyword=biblioauthorprogram]%
        \printbibliography[heading=nobibheading, section=1, env=countauthor,             keyword=biblioauthor]%
        \printbibliography[heading=nobibheading, section=1, env=countauthorvakscopuswos, filter=vakscopuswos]%
        \printbibliography[heading=nobibheading, section=1, env=countauthorscopuswos,    filter=scopuswos]%
        %
        \nocite{*}%
        %
        {\publications}
        
    По теме исследования опубликовано~X работ, из них~X работы в журналах WoS, Scopus, RSCI, а также в изданиях, рекомендованных для защиты в диссертационном совете МГУ имени М.В.~Ломоносова по специальности 05.13.18 и~X работ, опубликованных в иных изданиях.
    	
    	%, 3 работы в журналах, входящих в перечень изданий ВАК при Минобрнауки России
        %откомментить при использовании автоподсчета
%        Основные результаты по теме диссертации изложены в~\arabic{citeauthor}~печатных изданиях,
%        \arabic{citeauthorvak} из которых изданы в журналах, рекомендованных ВАК\sloppy%
%        \ifnum \value{citeauthorscopuswos}>0%
%            , \arabic{citeauthorscopuswos} "--- в~периодических научных журналах, индексируемых Web of~Science и Scopus\sloppy%
%        \fi%
%        \ifnum \value{citeauthorconf}>0%
%            , \arabic{citeauthorconf} "--- в~тезисах докладов.
%        \else%
%            .
%        \fi%
%        \ifnum \value{citeregistered}=1%
%            \ifnum \value{citeauthorpatent}=1%
%                Зарегистрирован \arabic{citeauthorpatent} патент.
%            \fi%
%            \ifnum \value{citeauthorprogram}=1%
%                Зарегистрирована \arabic{citeauthorprogram} программа для ЭВМ.
%            \fi%
%        \fi%
%        \ifnum \value{citeregistered}>1%
%            Зарегистрированы\ %
%            \ifnum \value{citeauthorpatent}>0%
%            \formbytotal{citeauthorpatent}{патент}{}{а}{}\sloppy%
%            \ifnum \value{citeauthorprogram}=0 . \else \ и~\fi%
%            \fi%
%            \ifnum \value{citeauthorprogram}>0%
%            \formbytotal{citeauthorprogram}{программ}{а}{ы}{} для ЭВМ.
%            \fi%
%        \fi%
        
        
        % К публикациям, в которых излагаются основные научные результаты диссертации на соискание учёной
        % степени, в рецензируемых изданиях приравниваются патенты на изобретения, патенты (свидетельства) на
        % полезную модель, патенты на промышленный образец, патенты на селекционные достижения, свидетельства
        % на программу для электронных вычислительных машин, базу данных, топологию интегральных микросхем,
        % зарегистрированные в установленном порядке.(в ред. Постановления Правительства РФ от 21.04.2016 N 335)
    \end{refsection}%
    \begin{refsection}[bl-author, bl-registered]
        % Это refsection=2.
        % Процитированные здесь работы:
        %  * попадают в авторскую библиографию, при usefootcite==0 и стиле `\insertbiblioauthorimportant`.
        %  * ни на что не влияют в противном случае

    \end{refsection}%
        %
        % Всё, что вне этих двух refsection, это refsection=0,
        %  * для диссертации - это нормальные ссылки, попадающие в обычную библиографию
        %  * для автореферата:
        %     * при usefootcite==0, ссылка корректно сработает только для источника из `external.bib`. Для своих работ --- напечатает "[0]" (и даже Warning не вылезет).
        %     * при usefootcite==1, ссылка сработает нормально. В авторской библиографии будут только процитированные в refsection=0 работы.
}


{\contribution} 

Все результаты работы получены автором лично под научным руководством д.ф.-м.н., проф. А.С. Крылова. В работах, написанных в соавторстве, вклад автора диссертации в полученные результаты математического моделирования, численные методы и разработку комплекса программ является определяющим.

{\defpositions}

\begin {enumerate}[beginpenalty=10000]
	\item Метод повышения разрешения изображений мигающей флуоресцентной микроскопии, основанный на итерационном процессе максимизации регуляризирующего функционала.
	
	\item Однопараметрический метод повышения резкости медицинских изображений на основе деформации пиксельной сетки.
	
	\item Метод автоматического контроля качества рентгенограмм грудной клетки для задач машинного обучения, основанный на анализе и классификации уровня жёсткости рентгеновского снимка грудной клетки с помощью порядковой регрессии.
	
	\item Программный комплекс повышения разрешения изображений мигающей флуоресцентной микроскопии, повышения резкости медицинских изображений, определения качества рентгенограмм грудной клетки, набор данных для решения задачи компьютерной диагностики туберкулёза.
\end {enumerate}


