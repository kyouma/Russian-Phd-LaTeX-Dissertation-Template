%%% Основные сведения %%%
\newcommand{\thesisAuthorLastName}{Пчелинцев}
\newcommand{\thesisAuthorOtherNames}{Яков Антонович}
\newcommand{\thesisAuthorInitials}{Я.\,А.}
\newcommand{\thesisAuthor}             % Диссертация, ФИО автора
{%
    \texorpdfstring{% \texorpdfstring takes two arguments and uses the first for (La)TeX and the second for pdf
        \thesisAuthorLastName~\thesisAuthorOtherNames% так будет отображаться на титульном листе или в тексте, где будет использоваться переменная
    }{%
        \thesisAuthorLastName, \thesisAuthorOtherNames% эта запись для свойств pdf-файла. В таком виде, если pdf будет обработан программами для сбора библиографических сведений, будет правильно представлена фамилия.
    }
}
\newcommand{\thesisAuthorShort}        % Диссертация, ФИО автора инициалами
{\thesisAuthorInitials~\thesisAuthorLastName}
%\newcommand{\thesisUdk}                % Диссертация, УДК
%{\fixme{xxx.xxx}}
\newcommand{\thesisTitle}              % Диссертация, название
{Математические методы адаптивного повышения качества биомедицинских изображений}
\newcommand{\thesisSpecialtyNumber}    % Диссертация, специальность, номер
{1.2.2}
\newcommand{\thesisSpecialtyTitle}     % Диссертация, специальность, название (название взято с сайта ВАК для примера)
{Математическое моделирование, численные методы и~комплексы программ}
%% \newcommand{\thesisSpecialtyTwoNumber} % Диссертация, вторая специальность, номер
%% {\fixme{XX.XX.XX}}
%% \newcommand{\thesisSpecialtyTwoTitle}  % Диссертация, вторая специальность, название
%% {\fixme{Теория и~методика физического воспитания, спортивной тренировки,
%% оздоровительной и~адаптивной физической культуры}}
\newcommand{\thesisDegree}             % Диссертация, ученая степень
{кандидата физико-математических наук}
\newcommand{\thesisDegreeShort}        % Диссертация, ученая степень, краткая запись
{канд. физ.-мат. наук}
\newcommand{\thesisCity}               % Диссертация, город написания диссертации
{Москва}
\newcommand{\thesisYear}               % Диссертация, год написания диссертации
{\the\year}
\newcommand{\thesisOrganization}       % Диссертация, организация
{Московский государственный университет имени~М.В.~Ломоносова}
\newcommand{\thesisOrganizationShort}  % Диссертация, краткое название организации для доклада
{\fixme{НазУчДисРаб}}

\newcommand{\thesisInOrganization}     % Диссертация, организация в предложном падеже: Работа выполнена в ...
{на кафедре математической физики факультета вычислительной математики и кибернетики Московского государственного университета имени~М.В.~Ломоносова}

%% \newcommand{\supervisorDead}{}           % Рисовать рамку вокруг фамилии
\newcommand{\supervisorFio}              % Научный руководитель, ФИО
{Крылов Андрей Серджевич}
\newcommand{\supervisorRegalia}          % Научный руководитель, регалии
{доктор физико-математических наук, профессор}
\newcommand{\supervisorFioShort}         % Научный руководитель, ФИО
{А.С.~Крылов}
\newcommand{\supervisorRegaliaShort}     % Научный руководитель, регалии
{д-р~физ.-мат.~наук,~проф.}

%% \newcommand{\supervisorTwoDead}{}        % Рисовать рамку вокруг фамилии
%% \newcommand{\supervisorTwoFio}           % Второй научный руководитель, ФИО
%% {\fixme{Фамилия Имя Отчество}}
%% \newcommand{\supervisorTwoRegalia}       % Второй научный руководитель, регалии
%% {\fixme{уч. степень, уч. звание}}
%% \newcommand{\supervisorTwoFioShort}      % Второй научный руководитель, ФИО
%% {\fixme{И.\,О.~Фамилия}}
%% \newcommand{\supervisorTwoRegaliaShort}  % Второй научный руководитель, регалии
%% {\fixme{уч.~ст.,~уч.~зв.}}

\newcommand{\opponentOneFio}           % Оппонент 1, ФИО
{Мухин Сергей Иванович}
\newcommand{\opponentOneRegalia}       % Оппонент 1, регалии
{доктор физико-математических наук, доцент}
\newcommand{\opponentOneJobPlace}      % Оппонент 1, место работы
{профессор кафедры вычислительных методов факультета вычислительной математики и кибернетики МГУ имени~М.В.~Ломоносова}
\newcommand{\opponentOneJobPost}       % Оппонент 1, должность
{\fixme{должность}}

\newcommand{\opponentTwoFio}           % Оппонент 2, ФИО
{Визильтер Юрий Валентинович}
\newcommand{\opponentTwoRegalia}       % Оппонент 2, регалии
{доктор физико-математических наук, старший научный сотрудник}
\newcommand{\opponentTwoJobPlace}      % Оппонент 2, место работы
{директор по направлению --- руководитель научного комплекса <<Искусственный интеллект и техническое зрение>>, Федеральное автономное учреждение <<Государственный научно-исследовательский институт авиационных систем>> (ФАУ~<<ГосНИИАС>>)}
\newcommand{\opponentTwoJobPost}       % Оппонент 2, должность
{\fixme{должность}}

\newcommand{\opponentThreeFio}         % Оппонент 3, ФИО
{Чернявский Алексей Станиславович}
\newcommand{\opponentThreeRegalia}     % Оппонент 3, регалии
{кандидат технических наук}
\newcommand{\opponentThreeJobPlace}    % Оппонент 3, место работы
{начальник отдела, департамент искусственного интеллекта, ООО~<<Исследовательский центр Самсунг>>}
\newcommand{\opponentThreeJobPost}     % Оппонент 3, должность
{\fixme{должность}}

%\newcommand{\leadingOrganizationTitle} % Ведущая организация, дополнительные строки. Удалить, чтобы не отображать в автореферате
%{\fixme{Федеральное государственное бюджетное образовательное учреждение высшего
%профессионального образования с~длинным длинным длинным длинным названием}}

\newcommand{\defenseDate}              % Защита, дата
{\fixme{DD mmmmmmmm YYYY~г.~в~XX часов}}
\newcommand{\defenseCouncilNumber}     % Защита, номер диссертационного совета
{МГУ.012.1}
\newcommand{\defenseCouncilTitle}      % Защита, учреждение диссертационного совета
{Московского государственного университета имени~М.В.~Ломоносова}
\newcommand{\defenseCouncilAddress}    % Защита, адрес учреждение диссертационного совета
{\fixme{119991, г. Москва, ул. Ленинские горы, д. 1, стр. 52, факультет ВМК, ауд. 685}}
\newcommand{\defenseCouncilPhone}      % Телефон для справок
{\fixme{+7~(0000)~00-00-00}}

\newcommand{\defenseSecretaryFio}      % Секретарь диссертационного совета, ФИО
{А.В.~Ильин}
\newcommand{\defenseSecretaryRegalia}  % Секретарь диссертационного совета, регалии
{доктор физико-математических наук, член-корреспондент РАН}            % Для сокращений есть ГОСТы, например: ГОСТ Р 7.0.12-2011 + http://base.garant.ru/179724/#block_30000

\newcommand{\synopsisLibrary}          % Автореферат, название библиотеки
{\fixme{научной библиотеки МГУ имени М.В. Ломоносова (Ломоносовский просп., д.~27)}}
\newcommand{\synopsisDate}             % Автореферат, дата рассылки
{"\rule{0.35cm}{0.15mm}" \fixme{\rule{1.5cm}{0.15mm}}  2023 г.}

% To avoid conflict with beamer class use \providecommand
\providecommand{\keywords}%            % Ключевые слова для метаданных PDF диссертации и автореферата
{}
