%% Согласно ГОСТ Р 7.0.11-2011:
%% 5.3.3 В заключении диссертации излагают итоги выполненного исследования, рекомендации, перспективы дальнейшей разработки темы.
%% 9.2.3 В заключении автореферата диссертации излагают итоги данного исследования, рекомендации и перспективы дальнейшей разработки темы.
\begin{enumerate}[beginpenalty=10000]
	\item Разработан итерационный регуляризирующий алгоритм повышения разрешения и резкости изображений флуоресцентной мигающей микроскопии.
	
	\item Найдены оптимальные функции смещения для деформационного метода повышения резкости изображений для трёх моделей ядер размытия, возникающих на практике. Предложен малопараметрический вариант алгоритма.
	
	\item Разработан нейросетевой метод контроля качества рентгеновских снимков грудной клетки, основанный на автоматическом анализе жёсткости рентгенограммы. Показана возможность применения метода для повышения точности компьютерной диагностики туберкулёза лёгких.
	
	\item Реализован программный комплекс, состоящий из модулей повышения разрешения изображений флуоресцентной мигающей микроскопии, повышения резкости медицинских изображений, определения качества рентгенограмм грудной клетки для задачи диагностики туберкулёза лёгких.
\end{enumerate}

