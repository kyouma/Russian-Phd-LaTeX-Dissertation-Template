%% Согласно ГОСТ Р 7.0.11-2011:
%% 5.3.3 В заключении диссертации излагают итоги выполненного исследования, рекомендации, перспективы дальнейшей разработки темы.
%% 9.2.3 В заключении автореферата диссертации излагают итоги данного исследования, рекомендации и перспективы дальнейшей разработки темы.
\begin {enumerate}[beginpenalty=10000]
	\item Разработан итерационный регуляризирующий алгоритм повышения разрешения и резкости изображений флуоресцентной мигающей микроскопии.
	
	\item Найдены оптимальные функции смещения для деформационного метода повышения резкости изображений для трёх видов ядер размытия, возникающих на практике. Предложены малопараметрические варианты алгоритма.
	
	\item Разработан и применён для повышения качества диагностики туберкулёза лёгких нейросетевой метод контроля качества рентгеновских снимков грудной клетки, основанный на автоматическом анализе жёсткости рентгенограммы.
	
	\item Реализован программный комплекс, состоящий из модулей повышения разрешения изображений флуоресцентной мигающей микроскопии, повышения резкости медицинских изображений, определения качества рентгенограмм грудной клетки для задачи диагностики туберкулёза лёгких и компьютерной диагностики туберкулёза лёгких.
\end {enumerate}

