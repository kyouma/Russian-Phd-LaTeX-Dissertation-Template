% Новые переменные, которые могут использоваться во всём проекте
% ГОСТ 7.0.11-2011
% 9.2 Оформление текста автореферата диссертации
% 9.2.1 Общая характеристика работы включает в себя следующие основные структурные
% элементы:
% актуальность темы исследования;
\newcommand{\actualityTXT}{Актуальность темы.}
% степень ее разработанности;
\newcommand{\progressTXT}{Степень разработанности темы.}
% цели и задачи;
\newcommand{\aimTXT}{Цели и задачи работы}
\newcommand{\tasksTXT}{задачи}
% научную новизну;
\newcommand{\noveltyTXT}{Научная новизна}
% теоретическую и практическую значимость работы;
\newcommand{\influenceTXT}{Теоретическая и практическая ценность}
% или чаще используют просто
%\newcommand{\influenceTXT}{Практическая значимость}
% методологию и методы исследования;
\newcommand{\methodsTXT}{Методология и методы исследования.}
% положения, выносимые на защиту;
\newcommand{\defpositionsTXT}{Основные положения, выносимые на~защиту}
% степень достоверности и апробацию результатов.
\newcommand{\reliabilityTXT}{Степень достоверности результатов}
\newcommand{\probationTXT}{Апробация работы}

\newcommand{\contributionTXT}{Личный вклад}
\newcommand{\publicationsTXT}{Публикации}
\newcommand{\structureandsizeTXT}{Структура и объем работы}

%%% Заголовки библиографии:

% для автореферата:
\newcommand{\bibtitleauthor}{Публикации автора по теме диссертации}

% для стиля библиографии `\insertbiblioauthorgrouped`
\newcommand{\bibtitleauthorvak}{В журналах, входящих в перечень изданий ВАК при Минобрнауки России:}
\newcommand{\bibtitleauthorscopus}{В изданиях, входящих в международную базу цитирования Scopus:}
\newcommand{\bibtitleauthorwos}{Научные статьи, опубликованные в журналах WoS, Scopus, RSCI, а также в изданиях, рекомендованных для защиты в диссертационном совете МГУ им. М.В. Ломоносова по специальности 1.2.2:}
\newcommand{\bibtitleauthorother}{В прочих изданиях:}
\newcommand{\bibtitleauthorconf}{Иные публикации:}
\newcommand{\bibtitleauthorpatent}{Зарегистрированные патенты:}
\newcommand{\bibtitleauthorprogram}{Зарегистрированные программы для ЭВМ:}

% для стиля библиографии `\insertbiblioauthorimportant`:
\newcommand{\bibtitleauthorimportant}{Наиболее значимые \protect\MakeLowercase\bibtitleauthor}

% для списка литературы в диссертации и списка чужих работ в автореферате:
\newcommand{\bibtitlefull}{Список литературы} % (ГОСТ Р 7.0.11-2011, 4)
